\documentclass[12pt]{article}

\usepackage[margin=0.5in]{geometry}
\usepackage{amsmath, amssymb}
\usepackage{hyperref}
\renewcommand{\arraystretch}{1.6}

\newcommand{\psetheader}[4]{%
  \noindent
  \begin{minipage}[t]{0.6\textwidth}
    #1\\
    #2\\
    #3
  \end{minipage}%
  \hfill
  \begin{minipage}[t]{0.35\textwidth}
    \raggedleft #4
  \end{minipage}

  \vspace{0.5\baselineskip}
  \hrule
  \vspace{1em}
}

\title{Aljabar Elementer - Diskusi Sesi 5}
\author{Anas Azhar \\ FST - Matematika - 056413438 \\ Universitas Terbuka}
\date{\today}


\begin{document}

\psetheader
  {\textit{Aljabar Elementer} - Diskusi Sesi 5}
  {Universitas Terbuka, FST - Matematika, 2025.1}
  {Anas Azhar (056413438)}
  {Sabtu, 8 November 2025}

\section*{Determinan dan OBE}
Dari sudut pandang geometri, determinan memiliki hubungan langsung dengan transformasi geometri. Misalkan sebuah persegi satuan di bidang koordinat dengan titik-titik sudutnya $ \left(0, 0\right) $,$ \left(1, 0\right) $,$ \left(0, 1\right) $, dan $ \left(1, 1\right) $. Sekaligus dilengkapi transformasi dengan matriks berikut.

\[
A = 
  \begin{pmatrix}
    2 & 1 \\ 1 & 1
  \end{pmatrix}
\]

\noindent Selanjutnya setiap titik mengalami operasi $ A \times titik $. Secara gambar, persegi akan bertransformasi menjadi jajar genjang dengan dua sisi vektor. Dan luas dari bentuk baru itu adalah determinannya.

\[
\det(A) = \left(2\right)\left(1\right) - \left(1\right)\left(1\right) = 1
\]
Artinya:

\begin{itemize}
    \item luas jajar genjang yang baru $1 \times L $
    \item dari informasi tadi, luasnya tetap namun bentuk berubah. 
\end{itemize}

\noindent Semisal perolehan determinan 2, artinya dua kali lipat semula, atau $ \dfrac{1}{2} $, luas setengah dari semula, dan jika $ \textbf{0} $ maka persegi berubah jadi garis dan tidak memiliki luas lagi.

\noindent Dengan OBE, penyederhanaan matriks menjadi lebih mudah dilakukan. Bentuk segitiga atas (upper triangular) dan perolehan diagonal menjadi jalan lain untuk mendapatkan nilai determinan sebuah matriks selain dengan cara kofaktor.

\[
\det(A) = \text{(hasil kali diagonal utamanya)}
\]

\noindent Aturan OBE seperti \textit{swap}, \textit{scale}, dan \textit{pivot} tidak akan merubah dan menghilangkan informasi determinan di sebuah matriks.

\section*{Soal 1 : Berikan contoh sebuah matriks $2 \times 2$, hitung determinannya menggunakan operasi baris elementer!}

Ambil contoh
\[
A = 
    \begin{pmatrix}
        \dfrac{1}{2} & 1\\[8pt]
        \dfrac{3}{4} & 2
    \end{pmatrix}
\]

\noindent Lakukan operasi baris $B_2 \leftarrow B_2 -\dfrac{3}{2} B_1$ sehingga
\[
\begin{pmatrix}
    \dfrac{1}{2} & 1\\[8pt]
    \dfrac{3}{4} & 2
\end{pmatrix}
\xrightarrow{B_2 - \dfrac{3}{2} B_1}
\begin{pmatrix}
    \dfrac{1}{2} & 1\\[8pt]
    0 & \dfrac{1}{2}
\end{pmatrix}
\]

\noindent Karena matriks segitiga atas, determinannya
\[
\det(A) = \left( \dfrac{1}{2} \right) \left(\dfrac{1}{2}\right) = \dfrac{1}{4}.
\]

\section*{Soal 2 : Dari matriks di atas hitung determinannya menggunakan uraian/kofaktor!}

\noindent Dengan rumus $ad-bc$ diperoleh
\[
\det = \left(\dfrac{1}{2}\right)(2) - (1)\left(\dfrac{3}{4}\right)
= 1 - \dfrac{3}{4}
= \dfrac{1}{4}
\]

\noindent Jadi determinannya yang dihitunh melalui dua cara menghasilkan nilai yang sama, yaitu $ \det(A) = \dfrac{1}{4} $.

\end{document}