\documentclass[12pt]{article}

\usepackage[margin=0.5in]{geometry}
\usepackage{amsmath, amssymb}
\usepackage{hyperref}
\renewcommand{\arraystretch}{1.6}

% Header pset untuk halaman pertama
\newcommand{\psetheader}[4]{%
  \noindent
  \begin{minipage}[t]{0.6\textwidth}
    #1\\
    #2\\
    #3
  \end{minipage}%
  \hfill
  \begin{minipage}[t]{0.35\textwidth}
    \raggedleft #4
  \end{minipage}

  \vspace{0.5\baselineskip}
  \hrule
  \vspace{1em}
}

\title{Matematika Dasar - Diskusi Sesi 5}
\author{Anas Azhar \ FST - Matematika - 056413438 \ Universitas Terbuka}
\date{\today}

\begin{document}

\psetheader
  {\textit{Matematika Dasar} - Diskusi Sesi 5}
  {Universitas Terbuka, FST - Matematika, 2025.1}
  {Anas Azhar (056413438)}
  {Minggu, 9 November 2025}

\section*{Relasi dan Sifat-sifatnya}

\begin{enumerate}
  \item Diketahui himpunan \(A = \{3,4\}\), \(B = \{2,4,5,6\}\).
        Buatlah relasi dalam bentuk pasangan terurut yang memenuhi:
        \begin{enumerate}
          \item \(R_1 = \{(a,b)\mid a \ge b;\ a\in A,\ b\in B\}\).
          \item \(R_2 = \{(a,b)\mid a + 1 \ge b;\ a\in A,\ b\in B\}\).
        \end{enumerate}

        \textbf{Penyelesaian:}

        Untuk \(R_1\):
        \[
          A = \{3,4\},\quad B = \{2,4,5,6\}.
        \]

        \begin{itemize}
          \item Untuk \(a=3\): syarat \(a \ge b\) berarti \(3 \ge b\).
                Dari \(B\) hanya \(b=2\) yang memenuhi.
                Maka \((3,2)\in R_1\).
          \item Untuk \(a=4\): syarat \(4 \ge b\).
                Dari \(B\), \(b=2\) dan \(b=4\) memenuhi.
                Maka \((4,2)\) dan \((4,4)\in R_1\).
        \end{itemize}

        Jadi
        \[
        \boxed{
            R_1 = \{(3,2),(4,2),(4,4)\}.
        }
        \]

        Untuk \(R_2\):
        \[
          R_2 = \{(a,b)\mid a+1 \ge b;\ a\in A,\ b\in B\}.
        \]

        \begin{itemize}
          \item Untuk \(a=3\): \(a+1=4\), sehingga \(b \le 4\).
                Dari \(B\): \(b=2,4\).
                Maka \((3,2)\) dan \((3,4)\in R_2\).
          \item Untuk \(a=4\): \(a+1=5\), sehingga \(b \le 5\).
                Dari \(B\): \(b=2,4,5\).
                Maka \((4,2)\), \((4,4)\), dan \((4,5)\in R_2\).
        \end{itemize}

        Jadi
        \[
        \boxed{
            R_2 = \{(3,2),(3,4),(4,2),(4,4),(4,5)\}.
        } 
        \]

  \item Diketahui \(\mathbb{Z}\) (himpunan bilangan bulat) dan relasi
        \[
          aRb \iff b = 3a,\quad a\in\mathbb{Z},\ b\in\mathbb{Z}.
        \]
        Tentukan domain \(D(R)\) dan range \(R(R)\).

        \textbf{Penyelesaian:}

        Relasi dapat ditulis sebagai
        \[
          R = \{(a,3a)\mid a\in\mathbb{Z}\}.
        \]

        \begin{itemize}
          \item Domain \(D(R)\) adalah semua elemen pertama dari setiap pasangan:
                untuk setiap \(a\in\mathbb{Z}\) ada \(b=3a\in\mathbb{Z}\),
                sehingga
                \[
                \boxed{
                    D(R) = \mathbb{Z}.
                }
                \]
          \item Range \(R(R)\) adalah semua elemen kedua dari setiap pasangan:
                semua berbentuk \(3a\) dengan \(a\in\mathbb{Z}\),
                yaitu kelipatan tiga,
                \[
                \boxed{
                    R(R) = \{3k \mid k\in\mathbb{Z}\}.
                }
                \]
        \end{itemize}

  \item Diketahui graf berarah dengan himpunan simpul
        \[
          S = \{1,2,3,4,5\}
        \]
        dan panah seperti pada gambar (2 mengarah ke 1 dan 3, 1 ke 4 dan 5, 5 ke 3, 3 ke 4).

        \begin{enumerate}
          \item Buatlah relasi dalam bentuk pasangan terurut.

                Dari arah panah didapat pasangan:
                \[
                \boxed{
                    R = \{(2,1),(1,4),(2,3),(1,5),(5,3),(3,4)\}.
                }
                \]

          \item Periksa apakah relasi \(R\) bersifat refleksif, simetris, dan transitif.

                \textbf{Refleksif:} \\
                Refleksif pada \(S\) berarti \((x,x)\in R\) untuk semua
                \(x\in S\).
                Dari graf tidak ada loop pada setiap simpul, sehingga
                tidak ada \((1,1),(2,2),(3,3),(4,4),(5,5)\) di dalam \(R\).
                Jadi \(R\) \emph{bukan} refleksif.

                \textbf{Simetris:} \\
                Simetris berarti jika \((a,b)\in R\) maka \((b,a)\in R\).
                Karena \((2,1)\in R\) tetapi \((1,2)\notin R\),
                maka \(R\) \emph{tidak} simetris.

                \textbf{Transitif:} \\
                Transitif berarti jika \((a,b)\in R\) dan \((b,c)\in R\)
                maka \((a,c)\in R\).
                Perhatikan bahwa \((2,1)\in R\) dan \((1,4)\in R\),
                sehingga jika \(R\) transitif harusnya \((2,4)\in R\).
                Namun \((2,4)\notin R\).
                Maka \(R\) \emph{tidak} transitif.
        \end{enumerate}
\end{enumerate}


\end{document}