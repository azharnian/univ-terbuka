\documentclass[12pt]{article}

\usepackage[margin=0.5in]{geometry}
\usepackage{amsmath, amssymb}

\usepackage{tikz}
\usepackage{pgfplots}
\pgfplotsset{compat=1.17}

\usepackage{hyperref}
\renewcommand{\arraystretch}{1.6}

\newcommand{\psetheader}[4]{%
  \noindent
  \begin{minipage}[t]{0.6\textwidth}
    #1\\
    #2\\
    #3
  \end{minipage}%
  \hfill
  \begin{minipage}[t]{0.35\textwidth}
    \raggedleft #4
  \end{minipage}

  \vspace{0.5\baselineskip}
  \hrule
  \vspace{1em}
}

\newcommand{\NamaMatkul}{Kalkulus Diferensial}
\newcommand{\NamaDiskusi}{Diskusi Sesi 7}
\newcommand{\NamaLengkap}{Anas Azhar}
\newcommand{\NIM}{056413438}
\newcommand{\FakultasProdi}{FST - Matematika}
\newcommand{\Kampus}{Universitas Terbuka}
\newcommand{\Term}{2025.1}
\newcommand{\TanggalTugas}{Minggu, 23 November 2025}

\title{\NamaMatkul{} - \NamaDiskusi}
\author{\NamaLengkap{} \\ \FakultasProdi{} - \NIM{} \\ \Kampus}
\date{\today}


\begin{document}

\psetheader
  {\textit{\NamaMatkul} - \NamaDiskusi}
  {\Kampus, \FakultasProdi, \Term}
  {\NamaLengkap{} (\NIM)}
  {\TanggalTugas}

\section*{Soal}
Tentukan asimtot tegak dan asimtot datar fungsi dengan persamaan
\[
xy^2 - 4y^2 - 16x = 0.
\]

\section*{Jawaban}

Mulai dari persamaan kurva:
\[
xy^2 - 4y^2 - 16x = 0.
\]

\noindent Kelompokkan suku-suku yang memuat $y^2$:
\[
xy^2 - 4y^2 - 16x = 0
\quad\Longrightarrow\quad
y^2(x-4) - 16x = 0.
\]

\noindent
Pindahkan $16x$ ke ruas kanan:
\[
y^2(x-4) = 16x.
\]

\noindent
Selama $x \neq 4$, dapat dibagi dengan $(x-4)$:
\[
y^2 = \frac{16x}{x-4}.
\]

\subsection*{Asimtot Tegak}

Asimtot tegak terjadi jika ketika $x$ mendekati suatu bilangan $a$, nilai $y$ menuju tak hingga. Dari
\[
y^2 = \frac{16x}{x-4},
\]
perhatikan perilaku saat $x \to 4$.

\noindent
Untuk $x \to 4$ diperoleh
\[
\frac{16x}{x-4} \to \pm\infty,
\]
sehingga
\[
y^2 \to \infty \quad\Longrightarrow\quad |y| \to \infty.
\]

\noindent
Jadi, $x = 4$ adalah asimtot tegak kurva tersebut.

\[
\boxed{\text{Asimtot tegak: } x = 4}
\]

\subsection*{Asimtot Datar}

Asimtot datar dicari dari limit $y$ saat $x \to \pm\infty$. Dari
\[
y^2 = \frac{16x}{x-4}
= 16 \cdot \frac{x}{x-4},
\]
Hitung limit saat $x \to \infty$:
\[
\lim_{x\to\infty} y^2
= 16 \cdot \lim_{x\to\infty} \frac{x}{x-4}
= 16 \cdot 1 = 16.
\]

\noindent
Maka
\[
\lim_{x\to\infty} y = \pm 4.
\]

\noindent
Artinya, grafik kurva mendekati dua garis horizontal $y = 4$ dan $y = -4$ ketika $x$ menuju tak hingga.

\[
\boxed{\text{Asimtot datar: } y = 4 \text{ dan } y = -4}
\]

\subsection*{Kesimpulan}

Asimtot-asimtot dari kurva dengan persamaan
\[
xy^2 - 4y^2 - 16x = 0
\]
adalah:
\[
\text{asimtot tegak: } x = 4, \qquad
\text{asimtot datar: } y = 4 \text{ dan } y = -4.
\]

\end{document}