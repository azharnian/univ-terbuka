\documentclass[12pt]{article}

\usepackage[margin=0.5in]{geometry}
\usepackage{amsmath, amssymb}
\usepackage{hyperref}
\renewcommand{\arraystretch}{1.6}

\newcommand{\psetheader}[4]{%
  \noindent
  \begin{minipage}[t]{0.6\textwidth}
    #1\\
    #2\\
    #3
  \end{minipage}%
  \hfill
  \begin{minipage}[t]{0.35\textwidth}
    \raggedleft #4
  \end{minipage}

  \vspace{0.5\baselineskip}
  \hrule
  \vspace{1em}
}

\newcommand{\NamaMatkul}{Kalkulus Diferensial}
\newcommand{\NamaDiskusi}{Diskusi Sesi 8}
\newcommand{\NamaLengkap}{Anas Azhar}
\newcommand{\NIM}{056413438}
\newcommand{\FakultasProdi}{FST - Matematika}
\newcommand{\Kampus}{Universitas Terbuka}
\newcommand{\Term}{2025.1}
\newcommand{\TanggalTugas}{Minggu, 30 November 2025}

\title{\NamaMatkul{} - \NamaDiskusi}
\author{\NamaLengkap{} \\ \FakultasProdi{} - \NIM{} \\ \Kampus}
\date{\today}


\begin{document}

\psetheader
  {\textit{\NamaMatkul} - \NamaDiskusi}
  {\Kampus, \FakultasProdi, \Term}
  {\NamaLengkap{} (\NIM)}
  {\TanggalTugas}


\section*{Soal 1 : 
}
Hitung nilai limit berikut:
\[
\lim_{x \to 0} \frac{\sin^2 x - \sin x}{x}.
\]


\section*{Jawaban}
\[
\text{Catatan: } \frac{d}{dx}(\sin^2 x) = \frac{d}{dx}((\sin x)^2)
= 2\sin x \cdot \cos x \quad (\text{aturan rantai})
\]

\[
\lim_{x\to 0} \frac{\sin^2 x - \sin x}{x}
\overset{0/0}{=}
\lim_{x\to 0} \frac{2\sin x\cos x - \cos x}{1}
=
\lim_{x\to 0} \cos x(2\sin x - 1)
=
1(2\cdot 0 - 1)
=
-1.
\]

\section*{Soal 2 : }

Tentukan nilai
\[
\lim_{x \to \infty} \frac{x^2 + 3x + 1}{2x^2 - 2}.
\]

\section*{Jawaban}
\[
\lim_{x\to\infty} \frac{x^2 + 3x + 1}{2x^2 - 2}
\overset{\infty/\infty}{=}
\lim_{x\to\infty} \frac{2x + 3}{4x}
=
\lim_{x\to\infty} \frac{2 + \frac{3}{x}}{4}
=
\frac{2 + 0}{4}
=
\frac{1}{2}.
\]

\end{document}