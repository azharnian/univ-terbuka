\documentclass[12pt]{article}

\usepackage[margin=0.5in]{geometry}
\usepackage{amsmath, amssymb}
\usepackage{hyperref}
\renewcommand{\arraystretch}{1.6}

\newcommand{\psetheader}[4]{%
  \noindent
  \begin{minipage}[t]{0.6\textwidth}
    #1\\
    #2\\
    #3
  \end{minipage}%
  \hfill
  \begin{minipage}[t]{0.35\textwidth}
    \raggedleft #4
  \end{minipage}

  \vspace{0.5\baselineskip}
  \hrule
  \vspace{1em}
}

\title{Aljabar Elementer - Tugas 2 Sesi 5}
\author{Anas Azhar \ FST - Matematika - 056413438 \ Universitas Terbuka}
\date{\today}

\begin{document}

\psetheader
  {\textit{Aljabar Elementer} - Tugas 2 Sesi 5}
  {Universitas Terbuka, FST - Matematika, 2025.1}
  {Anas Azhar (056413438)}
  {Sabtu, 8 November 2025}

  % ===========================SOAL1=============================
\section*{Soal 1}
Untuk nilai-nilai $a$ berapakah sistem berikut ini tidak memiliki jawaban, memiliki jawabn tunggal, dan memiliki jawaban banyak:

\[
\begin{aligned}
  x_1 + x_2 + x_3 &= 4,\\
  x_3 &= 2,\\
  (a^2 - 4)x_3 &= a - 2.
\end{aligned}
\]
\noindent [Poin 20]

\section*{Solusi :}
\noindent Penyelesaian dengan pendekatan Matriks
\noindent Bentuk $A \cdot X = B$

\[
\begin{bmatrix}
  1 & 1 & 1\\
  0 & 0 & 1\\
  0 & 0 & a^2 - 4
\end{bmatrix}
\cdot
\begin{bmatrix}
  x_1\\
  x_2\\
  x_3
\end{bmatrix}
=
\begin{bmatrix}
  4\\
  2\\
  a-2
\end{bmatrix}.
\]

\noindent Matriks koefisien
\[
A=
\begin{bmatrix}
  1 & 1 & 1\\
  0 & 0 & 1\\
  0 & 0 & a^2 - 4 
\end{bmatrix}
\]

\noindent Karena matriks ini berbentuk segitiga bawah, determinannya adalah
\[
\det(A) = 1 \times 0 \times (a^2 - 4) = 0.
\]
\noindent Artinya matriks $A$ \emph{singular}, sehingga sistem tidak dapat memiliki satu solusi unik.

\noindent Untuk menentukan apakah sistem masih konsisten, periksa
\[
[A|B] = 
\begin{bmatrix}
  1 & 1 & 1 & 4\\
  0 & 0 & 1 & 2\\
  0 & 0 & a^2 - 4 & a-2 
\end{bmatrix}
\]
\noindent Dari baris kedua, diperoleh $x_3 = 2$. Substitusi ke baris ketiga:
\[
(a^2 - 4)\cdot 2 = a-2
\Rightarrow 2a^2 -a -6 = 0.
\]

\noindent Persamaan ini memenuhi $a = 2$ atau $a = -\dfrac{3}{2}$.

\subsection*{Kasus 1: $a \neq 2$ dan $a \neq -\frac{3}{2}$}

Baris ketiga menyebabkan pertentangan sehingga
\[
\text{rank}(A) = 2, \quad \text{rank}([A|B]) = 3.
\]
Sistem tidak konsisten $\Rightarrow$ \textbf{tidak memiliki solusi}.

\subsection*{Kasus 2: $a = 2$ atau $a = -\frac{3}{2}$}

Baris ketiga menjadi nol seluruhnya, maka
\[
\text{rank}(A) = \text{rank}([A|B]) = 2 < 3.
\]
Sistem konsisten tetapi tak unik, sehingga memiliki
\textbf{tak hingga banyak solusi}.
Dari dua baris pertama:
\[
x_3 = 2, \quad x_1 + x_2 = 2.
\]

\subsection*{Kesimpulan}

\[
\begin{cases}
\text{Tidak ada solusi} & \text{jika } a \neq 2,\, -\dfrac{3}{2},\\[0.4em]
\text{Banyak solusi} & \text{jika } a = 2 \text{ atau } a = -\dfrac{3}{2},\\[0.4em]
\text{Tidak ada solusi tunggal karena } \det(A) = 0.
\end{cases}
\]

% ===========================SOAL2=============================
\section*{Soal 2}
Diketahui sistem persamaan linear
\[
\begin{aligned}
x + 2y - 3z &= 0,\\
3x - y + 5z &= 0,\\
4x + y + (k^2 - 14)z &= 0.
\end{aligned}
\]
Sistem memiliki jawaban tak hingga banyak. Tentukan nilai $k$.
\noindent [Poin 20]
\section*{Solusi : }
Sistem homogen ini akan memiliki jawaban tak hingga banyak apabila
matriks koefisiennya singular, yaitu $\det(A) = 0$.

\noindent Matriks koefisien:
\[
A = \begin{bmatrix}
1 & 2 & -3\\
3 & -1 & 5\\
4 & 1 & k^2 - 14
\end{bmatrix}.
\]

\noindent Dengan operasi baris elementer ubah $A$ menjadi matriks segitiga atas.

\[
\begin{aligned}
B_2 &\leftarrow B_2 - 3B_1,\\
B_3 &\leftarrow B_3 - 4B_1,
\end{aligned}
\]
sehingga diperoleh
\[
\begin{bmatrix}
1 & 2 & -3\\
0 & -7 & 14\\
0 & -7 & k^2 - 2
\end{bmatrix}.
\]

\noindent Kemudian
\[
B_3 \leftarrow B_3 - B_2
\]
\[
U =
\begin{bmatrix}
1 & 2 & -3\\
0 & -7 & 14\\
0 & 0 & k^2 - 16
\end{bmatrix},
\]
yang merupakan matriks segitiga atas. Maka
\[
\det(A) = \det(U)
= 1 \cdot (-7) \cdot (k^2 - 16)
= -7(k^2 - 16)
= -7(k - 4)(k + 4).
\]

\noindent Agar sistem memiliki jawaban tak hingga banyak,
\[
\det(A) = 0 \quad\Rightarrow\quad -7(k - 4)(k + 4) = 0.
\]
Karena $-7 \neq 0$, diperoleh
\[
(k - 4)(k + 4) = 0 \Rightarrow k = 4 \ \text{atau}\ k = -4.
\]

\noindent Jadi, nilai $k$ yang membuat sistem memiliki jawaban tak hingga banyak
adalah
\[
\boxed{k = 4 \ \text{atau}\ k = -4.}
\]

% ===========================SOAL3=============================
\section*{Soal 3}
Misalkan
\[
B=
\begin{bmatrix}
3 & -2 & 0\\
-2 & 3 & 0\\
0 & 0 & 5
\end{bmatrix}.
\]
Tentukan $\det(B)$ dengan menggunakan Operasi Baris Elementer.
\noindent [Poin 20]

\section*{Solusi}
Gunakan operasi baris elementer,
yaitu $B_i \leftarrow B_i + cB_j$.
\noindent Nolkan elemen di bawah pivot pertama pada kolom pertama dengan operasi
\[
B_2 \leftarrow B_2 + \frac{2}{3}B_1.
\]
Maka baris kedua menjadi
\[
[-2,\,3,\,0] + \tfrac{2}{3}[3,\,-2,\,0]
= [0,\tfrac{5}{3},0].
\]
\noindent Sehingga matriks berubah menjadi
\[
\begin{bmatrix}
3 & -2 & 0\\
0 & \tfrac{5}{3} & 0\\
0 & 0 & 5
\end{bmatrix},
\]
yang merupakan matriks segitiga atas. Diperoleh determinannya
\[
\det(B) = \det
\begin{bmatrix}
3 & -2 & 0\\
0 & \tfrac{5}{3} & 0\\
0 & 0 & 5
\end{bmatrix}.
\]

\noindent Untuk matriks segitiga atas, determinan sama dengan hasil kali elemen-elemen diagonal, sehingga
\[
\det(B) = 3 \cdot \frac{5}{3} \cdot 5 = 25.
\]

\noindent Jadi,
\[
\boxed{\det(B) = 25}.
\]

% ===========================SOAL4=============================
\section*{Soal 4}
Misalkan
\[
\left[\begin{array}{ccc|c}
a & 0 & b & 2\\
a & a & 4 & 4\\
0 & a & 2 & b
\end{array}\right]
\]
adalah matriks yang diperluas dari suatu sistem linier.
Tentukan nilai $a$ dan $b$ agar sistem tersebut memiliki solusi tunggal.
\noindent [Poin 20]

\section*{Solusi : }
Matriks koefisiennya adalah
\[
A =
\begin{bmatrix}
a & 0 & b\\
a & a & 4\\
0 & a & 2
\end{bmatrix}.
\]

\noindent $\det(A)$ dengan operasi baris elementer jenis
$B_i \leftarrow B_i + cB_j$.

\noindent Pertama,
\[
B_2 \leftarrow B_2 - B_1
\Rightarrow
\begin{bmatrix}
a & 0 & b\\
0 & a & 4-b\\
0 & a & 2
\end{bmatrix}.
\]

\noindent Kemudian,
\[
B_3 \leftarrow B_3 - B_2
\Rightarrow
U =
\begin{bmatrix}
a & 0 & b\\
0 & a & 4-b\\
0 & 0 & b-2
\end{bmatrix}.
\]

\noindent Matriks $U$ adalah segitiga atas, sehingga
\[
\det(A) = \det(U) = a \cdot a \cdot (b-2) = a^2(b-2).
\]

\noindent Agar sistem memiliki solusi tunggal, diperlukan
\[
\det(A) \neq 0 \quad \Rightarrow \quad a^2(b-2) \neq 0,
\]
sehingga
\[
a \neq 0 \quad \text{dan} \quad b \neq 2.
\]


\noindent Sistem memiliki solusi tunggal apabila
\[
\boxed{a \neq 0 \text{ dan } b \neq 2.}
\]

% ===========================SOAL5=============================
\section*{Soal 5}
Misalkan
\[
A =
\begin{bmatrix}
4 & 0 & 1\\
-2 & 1 & 0\\
-2 & 0 & 1
\end{bmatrix}.
\]
Tentukan $\det(A)$ dengan menggunakan ekspansi kofaktor.
\noindent [Poin 20]

\section*{Solusi}
Lakukan ekspansi menurut kolom kedua karena banyak elemen nol.

\[
\det(A)
= a_{12}C_{12} + a_{22}C_{22} + a_{32}C_{32}.
\]
Pada kolom kedua, hanya $a_{22} = 1$ yang bukan nol, sehingga
\[
\det(A) = a_{22} \cdot C_{22}.
\]

\noindent Hitung kofaktor $C_{22}$:
hapus baris ke-2 dan kolom ke-2 dari $A$,
\[
M_{22} =
\begin{vmatrix}
4 & 1\\
-2 & 1
\end{vmatrix}
= (4)(1) - (1)(-2) = 4 + 2 = 6.
\]
Karena $C_{22} = (-1)^{2+2}M_{22} = (+1)(6) = 6$, maka
\[
\det(A) = a_{22}C_{22} = 1 \times 6 = 6.
\]
\noindent Jadi,
\[
\boxed{\det(A) = 6.}
\]

\end{document}