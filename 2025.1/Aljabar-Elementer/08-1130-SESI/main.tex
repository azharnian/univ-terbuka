\documentclass[12pt]{article}

\usepackage[margin=0.5in]{geometry}
\usepackage{amsmath, amssymb}
\usepackage{hyperref}
\renewcommand{\arraystretch}{1.6}

\newcommand{\psetheader}[4]{%
  \noindent
  \begin{minipage}[t]{0.6\textwidth}
    #1\\
    #2\\
    #3
  \end{minipage}%
  \hfill
  \begin{minipage}[t]{0.35\textwidth}
    \raggedleft #4
  \end{minipage}

  \vspace{0.5\baselineskip}
  \hrule
  \vspace{1em}
}

\newcommand{\NamaMatkul}{Aljabar Elementer}
\newcommand{\NamaDiskusi}{Diskusi Sesi 8}
\newcommand{\NamaLengkap}{Anas Azhar}
\newcommand{\NIM}{056413438}
\newcommand{\FakultasProdi}{FST - Matematika}
\newcommand{\Kampus}{Universitas Terbuka}
\newcommand{\Term}{2025.1}
\newcommand{\TanggalTugas}{Minggu, 30 November 2025}

\title{\NamaMatkul{} - \NamaDiskusi}
\author{\NamaLengkap{} \\ \FakultasProdi{} - \NIM{} \\ \Kampus}
\date{\today}


\begin{document}

\psetheader
  {\textit{\NamaMatkul} - \NamaDiskusi}
  {\Kampus, \FakultasProdi, \Term}
  {\NamaLengkap{} (\NIM)}
  {\TanggalTugas}


\section*{Soal : Berikan contoh dua buah vektor di R3, lalu tentukan sudut antara keduanya dan kemudian, carilah perkalian silang dari kedua vektor tersebut.}

\section*{Jawaban}
 
Diberikan dua vektor di $\mathbb{R}^3$ sebagai berikut
\[
\vec{a} = \left\langle \frac{1}{2},\,\frac{1}{3},\,1 \right\rangle, 
\qquad
\vec{b} = \left\langle \frac{2}{3},\,1,\,-\frac{1}{2} \right\rangle.
\]
Tentukan:
\begin{enumerate}
  \item sudut antara $\vec{a}$ dan $\vec{b}$,
  \item hasil perkalian silang (cross product) $\vec{a} \times \vec{b}$.
\end{enumerate}

\bigskip

\noindent
\textbf{Penyelesaian.}

\textbf{1. Mencari sudut antara $\vec{a}$ dan $\vec{b}$.}

Pertama, kita hitung hasil kali titik (dot product) kedua vektor:
\begin{align*}
\vec{a} \cdot \vec{b}
&= \left(\frac{1}{2}\right)\left(\frac{2}{3}\right)
 + \left(\frac{1}{3}\right)(1)
 + (1)\left(-\frac{1}{2}\right) \\
&= \frac{2}{6} + \frac{1}{3} - \frac{1}{2} \\
&= \frac{1}{3} + \frac{1}{3} - \frac{1}{2} \\
&= \frac{2}{3} - \frac{1}{2} \\
&= \frac{4}{6} - \frac{3}{6} \\
&= \frac{1}{6}.
\end{align*}

Selanjutnya, kita hitung panjang (magnitudo) masing-masing vektor:
\begin{align*}
\lVert \vec{a} \rVert 
&= \sqrt{\left(\frac{1}{2}\right)^2 + \left(\frac{1}{3}\right)^2 + 1^2} \\
&= \sqrt{\frac{1}{4} + \frac{1}{9} + 1} \\
&= \sqrt{\frac{9}{36} + \frac{4}{36} + \frac{36}{36}} \\
&= \sqrt{\frac{49}{36}} 
= \frac{7}{6},
\end{align*}
\begin{align*}
\lVert \vec{b} \rVert
&= \sqrt{\left(\frac{2}{3}\right)^2 + 1^2 + \left(-\frac{1}{2}\right)^2} \\
&= \sqrt{\frac{4}{9} + 1 + \frac{1}{4}} \\
&= \sqrt{\frac{16}{36} + \frac{36}{36} + \frac{9}{36}} \\
&= \sqrt{\frac{61}{36}} 
= \frac{\sqrt{61}}{6}.
\end{align*}

Sudut $\theta$ antara $\vec{a}$ dan $\vec{b}$ memenuhi
\[
\cos\theta = \frac{\vec{a} \cdot \vec{b}}
                 {\lVert \vec{a} \rVert \,\lVert \vec{b} \rVert}.
\]

Substitusikan nilai yang telah diperoleh:
\begin{align*}
\cos\theta
&= \frac{\frac{1}{6}}
        {\left(\frac{7}{6}\right)\left(\frac{\sqrt{61}}{6}\right)} \\
&= \frac{\frac{1}{6}}{\frac{7\sqrt{61}}{36}} \\
&= \frac{1}{6} \cdot \frac{36}{7\sqrt{61}} \\
&= \frac{6}{7\sqrt{61}}.
\end{align*}

Jadi sudutnya adalah
\[
\theta = \cos^{-1}\left(\frac{6}{7\sqrt{61}}\right).
\]

(Jika ingin nilai aproksimasi dalam derajat, bisa dihitung dengan kalkulator.)

\bigskip

\textbf{2. Mencari perkalian silang $\vec{a} \times \vec{b}$ dengan metode ekspansi kofaktor.}

Dengan matriks
\[
\vec{a} \times \vec{b}
=
\begin{vmatrix}
\hat{i} & \hat{j} & \hat{k} \\
\frac{1}{2} & \frac{1}{3} & 1 \\
\frac{2}{3} & 1 & -\frac{1}{2}
\end{vmatrix}
\]

Kita menggunakan ekspansi kofaktor pada baris pertama:

\[
\vec{a} \times \vec{b}
=
(+1)\hat{i}
\begin{vmatrix}
\frac{1}{3} & 1 \\
1 & -\frac{1}{2}
\end{vmatrix}
\;
+ (-1)\hat{j}
\begin{vmatrix}
\frac{1}{2} & 1 \\
\frac{2}{3} & -\frac{1}{2}
\end{vmatrix}
\;
+ (+1)\hat{k}
\begin{vmatrix}
\frac{1}{2} & \frac{1}{3} \\
\frac{2}{3} & 1
\end{vmatrix}
\]

Sekarang kita hitung satu per satu:

Untuk komponen $\hat{i}$:
\[
(+1)\hat{i}
\left( \frac{1}{3}\left(-\frac{1}{2}\right) - (1)(1) \right)
=
\hat{i}\left( -\frac{1}{6} - 1 \right)
=
-\frac{7}{6}\hat{i}
\]

Untuk komponen $\hat{j}$:
\[
(-1)\hat{j}
\left(
\frac{1}{2}\left(-\frac{1}{2}\right)
-
(1)\left(\frac{2}{3}\right)
\right)
=
(-1)\hat{j}\left(
-\frac{1}{4} - \frac{2}{3}
\right)
\]

Ubah ke penyebut 12:
\[
-\frac{1}{4} - \frac{2}{3}
= -\frac{3}{12} - \frac{8}{12}
= -\frac{11}{12}
\]

Sehingga:
\[
-\hat{j}\left(-\frac{11}{12}\right)
=
+\frac{11}{12}\hat{j}
\]

Untuk komponen $\hat{k}$:
\[
(+1)\hat{k}
\left(
\frac{1}{2}(1) - \frac{1}{3}\left(\frac{2}{3}\right)
\right)
=
\hat{k}
\left(
\frac{1}{2} - \frac{2}{9}
\right)
=
\hat{k}
\left(
\frac{9}{18} - \frac{4}{18}
\right)
=
\frac{5}{18}\hat{k}
\]

Sehingga hasil akhirnya:
\[
\vec{a} \times \vec{b}
=
-\frac{7}{6}\hat{i}
+
\frac{11}{12}\hat{j}
+
\frac{5}{18}\hat{k}
=
\left\langle -\frac{7}{6},\,\frac{11}{12},\,\frac{5}{18} \right\rangle.
\]

\bigskip

\textbf{Jawaban singkat:}
\begin{itemize}
  \item Sudut antara $\vec{a}$ dan $\vec{b}$:
  \[
  \theta = \cos^{-1}\left(\frac{6}{7\sqrt{61}}\right).
  \]
  \item Perkalian silang kedua vektor:
  \[
  \vec{a} \times \vec{b}
  = \left\langle -\frac{7}{6},\,\frac{11}{12},\,\frac{5}{18} \right\rangle.
  \]
\end{itemize}


\end{document}