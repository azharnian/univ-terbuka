\documentclass[12pt,a4paper]{article}

\usepackage{setspace}
\usepackage{geometry}
\usepackage{parskip}

% --------------------------------------------------------------
% Setting dokumen
% --------------------------------------------------------------
\geometry{margin=1in}
\setstretch{1.3}

% --------------------------------------------------------------
% Paket tanggal Indonesia
% --------------------------------------------------------------
\usepackage[useregional]{datetime2}
\DTMlangsetup[bahasa]{ord=raise,monthyearsep={~}}
\newcommand{\indonesiandate}[1]{\DTMdisplaydate{#1}{-1}{-1}{bahasa}}

% --------------------------------------------------------------
% Judul dan Identitas
% --------------------------------------------------------------
\title{Tugas 2}
\author{Anas Azhar \\ Fakultas Sains dan Teknologi – Matematika – 056413438 \\ Universitas Terbuka \\ Pendidikan Agama Islam}
\date{\today}

% --------------------------------------------------------------
\begin{document}
% --------------------------------------------------------------

\maketitle

\section*{Soal 1}

\textbf{a. Pengertian hukum Allah SWT dan sifat-sifatnya yang relevan dengan kasus ini}

Hukum Allah SWT adalah ketentuan dan pedoman hidup yang bersumber dari wahyu, yaitu Al-Qur'an dan sunnah Nabi Muhammad SAW. Hukum ini mengatur hubungan manusia dengan Allah, sesama manusia, dan lingkungan. Pada kasus hiburan malam tanpa batas waktu yang menyebabkan kelalaian ibadah dan pergaulan bebas, sifat-sifat hukum Allah SWT yang relevan antara lain:

\begin{enumerate}
    \item \textbf{Universal (syumul)}: Berlaku untuk seluruh zaman dan kondisi, sehingga tetap menjadi rujukan ketika aturan manusia bertentangan dengan nilai moral.
    \item \textbf{Menjaga kemaslahatan (maslahah)}: Tujuannya menjaga agama, jiwa, akal, keturunan, dan harta. Hiburan malam berlebihan bertentangan dengan tujuan ini.
    \item \textbf{Konsisten dan tidak berubah (tsabat)}: Nilai dasarnya tetap, yaitu menjaga manusia dari kerusakan moral dan spiritual.
    \item \textbf{Memberi perlindungan} melalui aturan seperti menjaga salat dan menjauhi zina.
\end{enumerate}

\textbf{b. Mengapa seorang muslim wajib taat kepada hukum Allah SWT dalam situasi seperti ini}

\begin{enumerate}
    \item \textbf{Ketaatan kepada Allah lebih tinggi daripada aturan manusia}. Ketika peraturan desa membuka peluang kerusakan, muslim tetap berpegang pada firman Allah.
    \item \textbf{Hukum Allah menjaga manusia dari kerusakan jangka panjang}. Secara sosial dan ilmiah, hiburan malam berlebih menurunkan kesehatan, produktivitas, dan kontrol diri.
    \item \textbf{Muslim bertanggung jawab secara personal}. Ibadah seperti salat Subuh adalah disiplin moral yang membentuk karakter, bukan sekadar rutinitas.
\end{enumerate}

\newpage
\section*{Soal 2}

\textbf{a. Arti penting diutusnya Nabi Muhammad SAW dalam menyampaikan hukum Allah SWT}

Diutusnya Nabi Muhammad SAW adalah bentuk petunjuk langsung agar manusia tidak salah memahami wahyu. Beliau:

\begin{enumerate}
    \item \textbf{Menjadi penjelas (mubayyin)} Al-Qur'an.
    \item \textbf{Menjadi teladan (uswah hasanah)} sehingga ajaran moral dapat dipraktikkan, bukan hanya dipahami.
    \item \textbf{Menghubungkan wahyu dengan realitas}, sehingga nilai agama dapat diterapkan dalam kehidupan manusia sepanjang sejarah.
\end{enumerate}

\textbf{b. Kedudukan Nabi Muhammad SAW terhadap hukum-hukum Al-Qur'an dan relevansi ajarannya}

\begin{enumerate}
    \item Nabi adalah \textbf{penyampai dan penafsir wahyu}. Sunnah menjelaskan ayat-ayat yang membutuhkan penjabaran.
    \item Ajaran moral Nabi bersifat \textbf{universal dan timeless}. Prinsip seperti kejujuran, amanah, keadilan, dan kasih sayang tidak dapat digantikan oleh teknologi.
    \item Nabi merupakan \textbf{sumber hukum kedua setelah Al-Qur'an}. Teknologi hanya alat; ia tidak memiliki nilai moral, sehingga ajaran Nabi tetap relevan sepanjang masa.
\end{enumerate}

\newpage
\section*{Soal 3}

\textbf{a. Mengapa agama dipandang sebagai sumber moral}

\begin{enumerate}
    \item \textbf{Memberikan standar benar dan salah yang stabil}. Moral berbasis kebiasaan sosial dapat berubah dan menyimpang, namun agama memberikan pedoman tetap.
    \item \textbf{Membangun integritas personal}. Kesadaran bahwa manusia diawasi Allah mencegah perilaku buruk bahkan di lingkungan yang rusak.
    \item \textbf{Didukung kajian ilmiah}. Nilai spiritual terbukti meningkatkan empati, kontrol diri, dan perilaku prososial.
\end{enumerate}

\textbf{b. Contoh akhlak mulia dalam menghadapi praktik korupsi di tempat kerja}

\begin{enumerate}
    \item \textbf{Menjaga kejujuran dan menolak terlibat korupsi} meskipun mendapat tekanan.
    \item \textbf{Bersabar, profesional, dan tetap tegas pada nilai moral}, tanpa membalas keburukan dengan keburukan.
\end{enumerate}

% --------------------------------------------------------------
\end{document}
% --------------------------------------------------------------