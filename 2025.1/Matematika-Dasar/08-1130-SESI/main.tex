\documentclass[12pt]{article}

\usepackage[margin=0.5in]{geometry}
\usepackage{amsmath, amssymb}
\usepackage{hyperref}
\renewcommand{\arraystretch}{1.6}

\newcommand{\psetheader}[4]{%
  \noindent
  \begin{minipage}[t]{0.6\textwidth}
    #1\\
    #2\\
    #3
  \end{minipage}%
  \hfill
  \begin{minipage}[t]{0.35\textwidth}
    \raggedleft #4
  \end{minipage}

  \vspace{0.5\baselineskip}
  \hrule
  \vspace{1em}
}

\newcommand{\NamaMatkul}{Matematika Dasar}
\newcommand{\NamaDiskusi}{Diskusi Sesi 8}
\newcommand{\NamaLengkap}{Anas Azhar}
\newcommand{\NIM}{056413438}
\newcommand{\FakultasProdi}{FST - Matematika}
\newcommand{\Kampus}{Universitas Terbuka}
\newcommand{\Term}{2025.1}
\newcommand{\TanggalTugas}{Minggu, 30 November 2025}

\title{\NamaMatkul{} - \NamaDiskusi}
\author{\NamaLengkap{} \\ \FakultasProdi{} - \NIM{} \\ \Kampus}
\date{\today}


\begin{document}

\psetheader
  {\textit{\NamaMatkul} - \NamaDiskusi}
  {\Kampus, \FakultasProdi, \Term}
  {\NamaLengkap{} (\NIM)}
  {\TanggalTugas}


\section*{Soal : Tunjukkan bahwa jika $n$ bilangan ganjil maka $n^2$ juga bilangan ganjil.
}

\section*{Pembuktian} 

\textbf{Bukti dengan kontrapositif.}
Misalkan $P$ adalah pernyataan ``$n$ bilangan ganjil'' dan $Q$ adalah pernyataan
``$n^2$ bilangan ganjil''. Pernyataan yang akan dibuktikan adalah
\[
P \to Q.
\]
Kontrapositif dari pernyataan tersebut adalah
\[
\neg Q \to \neg P,
\]
yang dalam konteks bilangan bulat dapat ditulis sebagai:
\[
\text{Jika } n^2 \text{ genap, maka } n \text{ genap.}
\]

\noindent
Sekarang dibuktikan kontrapositif tersebut. Misalkan $n^2$ genap. Maka terdapat
bilangan bulat $k$ sehingga
\[
n^2 = 2k.
\]
Akan ditunjukkan bahwa $n$ juga genap. Andaikan sebaliknya bahwa $n$ ganjil.
Jika $n$ ganjil, maka terdapat bilangan bulat $m$ sehingga
\[
n = 2m + 1.
\]
Dengan demikian
\[
n^2 = (2m+1)^2 = 4m^2 + 4m + 1 = 2(2m^2 + 2m) + 1,
\]
yang merupakan bilangan ganjil. Hal ini bertentangan dengan asumsi bahwa $n^2$
adalah bilangan genap. Jadi asumsi bahwa $n$ ganjil haruslah salah. Maka $n$
bukan bilangan ganjil, sehingga $n$ adalah bilangan genap.

\noindent
Kontrapositif dari pernyataan
``jika $n$ bilangan ganjil maka $n^2$ bilangan ganjil'' telah terbukti benar.
Oleh karena pernyataan dan kontrapositifnya ekuivalen secara logis, pernyataan
asli
\[
\text{Jika $n$ bilangan ganjil maka $n^2$ juga bilangan ganjil}
\]
telah terbukti.

\section*{Soal 2 : }
Buktikan menggunakan induksi matematika bahwa untuk $n \ge 2$ berlaku
\[
\frac{1}{2\cdot1} + \frac{1}{3\cdot2} + \frac{1}{4\cdot3} + \dots + \frac{1}{n(n-1)}
= \frac{n-1}{n}.
\]

\section*{Pembuktian}

Bentuk umum penjumlahan sebagai
\[
\sum_{k=2}^{n} \frac{1}{k(k-1)} = \frac{n-1}{n}, \quad n \ge 2.
\]

\noindent
\emph{Langkah 1 : Basis induksi:} untuk $n=2$,
\[
\sum_{k=2}^{2} \frac{1}{k(k-1)} = \frac{1}{2\cdot1} = \frac{1}{2},
\]
dan
\[
\frac{n-1}{n}\Big|_{n=2} = \frac{2-1}{2} = \frac{1}{2}.
\]
Jadi pernyataan benar untuk $n=2$.

\noindent
\emph{Langkah 2 : Induksi atau asumsikan untuk n = k (bilangan lain yg memenuhi)} dibuktikan bahwa

\noindent
Hipotesis induksi atau Base Case: andaikan untuk suatu $n \ge 2$ berlaku
\[
\sum_{k=2}^{n} \frac{1}{k(k-1)} = \frac{n-1}{n}.
\]

\noindent
\emph{Langkah 3 : Dengan menggunakan hipotesis induksi} 

\[
\sum_{k=2}^{n+1} \frac{1}{k(k-1)} = \frac{n}{n+1}.
\]

\[
\begin{aligned}
\sum_{k=2}^{n+1} \frac{1}{k(k-1)}
&= \sum_{k=2}^{n} \frac{1}{k(k-1)} + \frac{1}{(n+1)n} \\
&= \frac{n-1}{n} + \frac{1}{n(n+1)} \\
&= \frac{(n-1)(n+1)}{n(n+1)} + \frac{1}{n(n+1)} \\
&= \frac{n^2 - 1 + 1}{n(n+1)} \\
&= \frac{n^2}{n(n+1)} = \frac{n}{n+1}.
\end{aligned}
\]
Jadi pernyataan benar untuk $n+1$.

\noindent
Karena pernyataan benar untuk $n=2$ dan kebenarannya untuk $n$ 
mengakibatkan kebenarannya untuk $n+1$, maka dengan induksi matematika
diperoleh bahwa
\[
\sum_{k=2}^{n} \frac{1}{k(k-1)} = \frac{n-1}{n}
\]
berlaku untuk semua $n \ge 2$.


\section*{Soal 3 : Dengan menggunakan metode kontradiksi, buktikan bahwa
untuk $n$ bilangan asli, $n^2 + n + 1$ adalah bilangan ganjil.}

\section*{Pembuktian}
Misalkan
\[
P : \quad n^2 + n + 1 \text{ adalah bilangan ganjil}.
\]
\noindent
dibuktikan $P$ benar dengan metode kontradiksi. Andaikan sebaliknya, yaitu diasumsikan $\neg P$:
\[
\neg P : \quad n^2 + n + 1 \text{ adalah bilangan genap}.
\]
\noindent
Karena bilangan ganjil ditambah $1$ menghasilkan bilangan genap,
maka dari asumsi $\neg P$ diperoleh
\[
n^2 + n = \text{ganjil}. \qquad (1)
\]

\noindent
Namun,
\[
n^2 + n = n(n+1).
\]
Karena $n$ dan $(n+1)$ adalah dua bilangan asli berurutan, maka salah satu di antaranya pasti genap. 
Hasil kali bilangan genap dengan bilangan bulat menghasilkan bilangan genap, sehingga
\[
n^2 + n = n(n+1) = \text{genap}. \qquad (2)
\]

\noindent
Sekarang perhatikan (1) dan (2):
\[
n^2+n = \text{ganjil} \quad \land \quad n^2+n = \text{genap}.
\]
Ini merupakan kontradiksi logis, yaitu
\[
P \land \neg P,
\]
yang secara logika bernilai salah.

\noindent
Karena asumsi $\neg P$ menghasilkan kontradiksi, maka asumsi tersebut salah.
Dengan demikian $P$ benar, yaitu
\[
n^2 + n + 1 \text{ adalah bilangan ganjil}.
\]


\end{document}