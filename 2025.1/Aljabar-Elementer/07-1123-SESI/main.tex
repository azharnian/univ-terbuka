\documentclass[12pt]{article}

\usepackage[margin=0.5in]{geometry}
\usepackage{amsmath, amssymb}
\usepackage{hyperref}
\renewcommand{\arraystretch}{1.6}

\newcommand{\psetheader}[4]{%
  \noindent
  \begin{minipage}[t]{0.6\textwidth}
    #1\\
    #2\\
    #3
  \end{minipage}%
  \hfill
  \begin{minipage}[t]{0.35\textwidth}
    \raggedleft #4
  \end{minipage}

  \vspace{0.5\baselineskip}
  \hrule
  \vspace{1em}
}

\newcommand{\NamaMatkul}{Aljabar Elementer}
\newcommand{\NamaDiskusi}{Diskusi Sesi 7}
\newcommand{\NamaLengkap}{Anas Azhar}
\newcommand{\NIM}{056413438}
\newcommand{\FakultasProdi}{FST - Matematika}
\newcommand{\Kampus}{Universitas Terbuka}
\newcommand{\Term}{2025.1}
\newcommand{\TanggalTugas}{Minggu, 23 November 2025}

\title{\NamaMatkul{} - \NamaDiskusi}
\author{\NamaLengkap{} \\ \FakultasProdi{} - \NIM{} \\ \Kampus}
\date{\today}


\begin{document}

\psetheader
  {\textit{\NamaMatkul} - \NamaDiskusi}
  {\Kampus, \FakultasProdi, \Term}
  {\NamaLengkap{} (\NIM)}
  {\TanggalTugas}


\section*{Soal 1}
Berikan contoh sebuah himpunan vektor di $\mathbb{R}^2$, kemudian buktikan apakah himpunan tersebut bebas linear atau tidak.


\section*{Jawaban}

\subsection*{Himpunan vektor yang \textit{bebas linear} di $\mathbb{R}^2$}

Ambil himpunan vektor
\[
S = \left\{
v_1 =
\begin{pmatrix}
\frac{1}{2} \\[4pt]
\frac{2}{3}
\end{pmatrix},
\quad
v_2 =
\begin{pmatrix}
\frac{3}{4} \\[4pt]
\frac{1}{6}
\end{pmatrix}
\right\}
\subset \mathbb{R}^2.
\]

\noindent Menurut definisi, himpunan $S$ dikatakan \textit{bebas linear} jika satu-satunya solusi dari SPL homogen
\[
k_1 v_1 + k_2 v_2 = \mathbf{0}
\]
adalah $k_1 = 0$ dan $k_2 = 0$.

\noindent Tulis persamaan $k_1 v_1 + k_2 v_2 = \mathbf{0}$ dalam bentuk komponen:
\[
k_1
\begin{pmatrix}
\frac{1}{2} \\[4pt]
\frac{2}{3}
\end{pmatrix}
+
k_2
\begin{pmatrix}
\frac{3}{4} \\[4pt]
\frac{1}{6}
\end{pmatrix}
=
\begin{pmatrix}
0 \\[2pt]
0
\end{pmatrix}.
\]

\noindent Ini ekuivalen dengan sistem persamaan linear:
\[
\begin{cases}
\displaystyle \frac{1}{2}k_1 + \frac{3}{4}k_2 = 0,\\[6pt]
\displaystyle \frac{2}{3}k_1 + \frac{1}{6}k_2 = 0.
\end{cases}
\]

\noindent Sistem tersebut dapat ditulis dalam bentuk matriks:
\[
\begin{bmatrix}
\frac{1}{2} & \frac{3}{4} \\
\frac{2}{3} & \frac{1}{6}
\end{bmatrix}
\begin{bmatrix}
k_1 \\[2pt]
k_2
\end{bmatrix}
=
\begin{bmatrix}
0 \\[2pt]
0
\end{bmatrix}.
\]

\noindent OBE (Operasi Baris Elementer) pada matriks koefisien (bentuk diperluas dengan ruas kanan nol):

\[
\left[
\begin{array}{cc|c}
\frac{1}{2} & \frac{3}{4} & 0 \\
\frac{2}{3} & \frac{1}{6} & 0
\end{array}
\right].
\]

\noindent Hilangkan elemen di bawah pivot pertama $\frac{1}{2}$:
\[
R_2 \leftarrow R_2 - \frac{\frac{2}{3}}{\frac{1}{2}} R_1
= R_2 - \frac{4}{3} R_1.
\]

\noindent Hitung baris kedua yang baru:
\[
\frac{2}{3} - \frac{4}{3}\cdot \frac{1}{2}
= \frac{2}{3} - \frac{2}{3} = 0,
\]
\[
\frac{1}{6} - \frac{4}{3}\cdot \frac{3}{4}
= \frac{1}{6} - 1
= -\frac{5}{6},
\]
dan ruas kanan tetap $0$:
\[
0 - \frac{4}{3}\cdot 0 = 0.
\]

\noindent Sehingga matriksnya menjadi:
\[
\left[
\begin{array}{cc|c}
\frac{1}{2} & \frac{3}{4} & 0 \\
0 & -\frac{5}{6} & 0
\end{array}
\right].
\]

\noindent Dari baris kedua:
\[
-\frac{5}{6}k_2 = 0 \quad \Rightarrow \quad k_2 = 0.
\]
Substitusi ke baris pertama:
\[
\frac{1}{2}k_1 + \frac{3}{4}\cdot 0 = 0
\quad \Rightarrow \quad k_1 = 0.
\]

\noindent Jadi satu-satunya solusi SPL homogen ini adalah
\[
k_1 = 0, \quad k_2 = 0.
\]

\noindent Artinya \textbf{tidak ada solusi non-trivial} (tidak ada solusi dengan $(k_1,k_2) \neq (0,0)$). 

\medskip
\noindent
Dengan demikian, berdasarkan definisi,
\[
S = \left\{v_1, v_2\right\}
\text{ adalah himpunan vektor yang \textbf{bebas linear} di } \mathbb{R}^2.
\]

\section*{Soal 2}
Berikan contoh sebuah himpunan vektor yang membentang $\mathbb{R}^3$

\section*{Jawaban}

Ambil tiga vektor di $\mathbb{R}^3$ berikut:
\[
a =
\begin{pmatrix}
\frac{1}{2} \\[4pt]
0 \\[4pt]
0
\end{pmatrix},
\quad
b =
\begin{pmatrix}
\frac{1}{3} \\[4pt]
\frac{2}{3} \\[4pt]
0
\end{pmatrix},
\quad
c =
\begin{pmatrix}
\frac{1}{4} \\[4pt]
\frac{1}{2} \\[4pt]
\frac{3}{4}
\end{pmatrix}.
\]

\noindent Tunjukkan bahwa $\{a,b,c\}$ membentang $\mathbb{R}^3$, artinya untuk setiap
\[
u =
\begin{pmatrix}
x \\[2pt]
y \\[2pt]
z
\end{pmatrix}
\in \mathbb{R}^3
\]
ada skalar $k_1,k_2,k_3 \in \mathbb{R}$ sehingga
\[
k_1 a + k_2 b + k_3 c = u.
\]

\[
k_1
\begin{pmatrix}
\frac{1}{2} \\[4pt]
0 \\[4pt]
0
\end{pmatrix}
+
k_2
\begin{pmatrix}
\frac{1}{3} \\[4pt]
\frac{2}{3} \\[4pt]
0
\end{pmatrix}
+
k_3
\begin{pmatrix}
\frac{1}{4} \\[4pt]
\frac{1}{2} \\[4pt]
\frac{3}{4}
\end{pmatrix}
=
\begin{pmatrix}
x \\[4pt]
y \\[4pt]
z
\end{pmatrix}.
\]

\noindent Ekuivalen dengan sistem persamaan linear:
\[
\begin{cases}
\displaystyle \frac{1}{2}k_1 + \frac{1}{3}k_2 + \frac{1}{4}k_3 = x,\\[6pt]
\displaystyle 0\cdot k_1 + \frac{2}{3}k_2 + \frac{1}{2}k_3 = y,\\[6pt]
\displaystyle 0\cdot k_1 + 0\cdot k_2 + \frac{3}{4}k_3 = z.
\end{cases}
\]

\noindent Bentuk matriks diperluas:
\[
\left[
\begin{array}{ccc|c}
\frac{1}{2} & \frac{1}{3} & \frac{1}{4} & x \\
0 & \frac{2}{3} & \frac{1}{2} & y \\
0 & 0 & \frac{3}{4} & z
\end{array}
\right].
\]

\noindent Matriks koefisien:
\[
A =
\begin{bmatrix}
\frac{1}{2} & \frac{1}{3} & \frac{1}{4} \\
0 & \frac{2}{3} & \frac{1}{2} \\
0 & 0 & \frac{3}{4}
\end{bmatrix}
\]

\noindent Dengan diagonal utama
\[
\frac{1}{2},\ \frac{2}{3},\ \frac{3}{4},
\]
yang semuanya \emph{tidak nol}. Ini berarti $\det(A) = \frac{1}{2}\cdot\frac{2}{3}\cdot\frac{3}{4} \neq 0$, sehingga matriks $A$ invertibel dan sistem selalu memiliki solusi tunggal untuk setiap $(x,y,z)$.

\noindent Menyelesaikan sistem ini secara eksplisit dengan substitusi balik.

\paragraph{Langkah 1: Menentukan $k_3$.}
Dari persamaan ketiga:
\[
\frac{3}{4}k_3 = z
\quad\Rightarrow\quad
k_3 = \frac{4}{3}z.
\]

\paragraph{Langkah 2: Menentukan $k_2$.}
Dari persamaan kedua:
\[
\frac{2}{3}k_2 + \frac{1}{2}k_3 = y.
\]
Substitusikan $k_3 = \frac{4}{3}z$:
\[
\frac{2}{3}k_2 + \frac{1}{2}\cdot\frac{4}{3}z = y
\quad\Rightarrow\quad
\frac{2}{3}k_2 + \frac{2}{3}z = y.
\]
Maka
\[
\frac{2}{3}k_2 = y - \frac{2}{3}z
\quad\Rightarrow\quad
k_2 = \frac{3}{2}y - z.
\]

\paragraph{Langkah 3: Menentukan $k_1$.}
Dari persamaan pertama:
\[
\frac{1}{2}k_1 + \frac{1}{3}k_2 + \frac{1}{4}k_3 = x.
\]
Substitusikan $k_2 = \frac{3}{2}y - z$ dan $k_3 = \frac{4}{3}z$:
\[
\frac{1}{3}k_2 = \frac{1}{3}\left(\frac{3}{2}y - z\right)
= \frac{1}{2}y - \frac{1}{3}z,
\]
\[
\frac{1}{4}k_3 = \frac{1}{4}\cdot \frac{4}{3}z = \frac{1}{3}z.
\]
Sehingga
\[
\frac{1}{3}k_2 + \frac{1}{4}k_3
= \left(\frac{1}{2}y - \frac{1}{3}z\right) + \frac{1}{3}z
= \frac{1}{2}y.
\]
Maka persamaan pertama menjadi:
\[
\frac{1}{2}k_1 + \frac{1}{2}y = x
\quad\Rightarrow\quad
\frac{1}{2}k_1 = x - \frac{1}{2}y
\quad\Rightarrow\quad
k_1 = 2x - y.
\]

\noindent
Untuk setiap
\[
u =
\begin{pmatrix}
x \\ y \\ z
\end{pmatrix}
\in \mathbb{R}^3,
\]
diperoleh
\[
\boxed{
\begin{aligned}
k_1 &= 2x - y,\\[4pt]
k_2 &= \frac{3}{2}y - z,\\[4pt]
k_3 &= \frac{4}{3}z.
\end{aligned}
}
\]

\noindent Artinya, untuk setiap $u \in \mathbb{R}^3$ terdapat skalar $k_1,k_2,k_3$ (yang merupakan bilangan real, dan berasal dari perhitungan dengan koefisien rasional) sedemikian sehingga
\[
k_1 a + k_2 b + k_3 c = u.
\]

\noindent Dengan demikian, himpunan vektor
\[
\left\{a,b,c\right\}
=
\left\{
\begin{pmatrix}
\frac{1}{2} \\[2pt] 0 \\[2pt] 0
\end{pmatrix},
\begin{pmatrix}
\frac{1}{3} \\[2pt] \frac{2}{3} \\[2pt] 0
\end{pmatrix},
\begin{pmatrix}
\frac{1}{4} \\[2pt] \frac{1}{2} \\[2pt] \frac{3}{4}
\end{pmatrix}
\right\}
\]
\textbf{membentang} $\mathbb{R}^3$, karena kombinasi linear $k_1 a + k_2 b + k_3 c$ dapat menghasilkan setiap vektor $u \in \mathbb{R}^3$.

\end{document}