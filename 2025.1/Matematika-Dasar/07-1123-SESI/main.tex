\documentclass[12pt]{article}

\usepackage[margin=0.5in]{geometry}
\usepackage{amsmath, amssymb}
\usepackage{hyperref}
\renewcommand{\arraystretch}{1.6}

\newcommand{\psetheader}[4]{%
  \noindent
  \begin{minipage}[t]{0.6\textwidth}
    #1\\
    #2\\
    #3
  \end{minipage}%
  \hfill
  \begin{minipage}[t]{0.35\textwidth}
    \raggedleft #4
  \end{minipage}

  \vspace{0.5\baselineskip}
  \hrule
  \vspace{1em}
}

\newcommand{\NamaMatkul}{Matematika Dasar}
\newcommand{\NamaDiskusi}{Diskusi Sesi 7}
\newcommand{\NamaLengkap}{Anas Azhar}
\newcommand{\NIM}{056413438}
\newcommand{\FakultasProdi}{FST - Matematika}
\newcommand{\Kampus}{Universitas Terbuka}
\newcommand{\Term}{2025.1}
\newcommand{\TanggalTugas}{Minggu, 23 November 2025}

\title{\NamaMatkul{} - \NamaDiskusi}
\author{\NamaLengkap{} \\ \FakultasProdi{} - \NIM{} \\ \Kampus}
\date{\today}


\begin{document}

\psetheader
  {\textit{\NamaMatkul} - \NamaDiskusi}
  {\Kampus, \FakultasProdi, \Term}
  {\NamaLengkap{} (\NIM)}
  {\TanggalTugas}


\section*{Soal 1. Misalkan $B = \mathbb{N}$ dan $C = \{1\} \cup \{2n \mid n \in \mathbb{N}\}$. Tunjukkan bahwa $B \preceq C$.}

\section*{Jawaban}

\subsection*{Membuktikan bahwa $B \preceq C$}

Diketahui:
\[
B = \mathbb{N} = \{1,2,3,4,\dots\}
\]
\[
C = \{1\} \cup \{2n \mid n \in \mathbb{N}\} = \{1,2,4,6,8,\dots\}.
\]

Untuk menunjukkan bahwa $B \preceq C$, cukup dibuktikan bahwa ada fungsi injektif
\[
f : B \rightarrow C.
\]

Definisikan fungsi:
\[
f(n) = 2n.
\]

Jelas bahwa $f(n) \in C$, karena semua bilangan genap merupakan anggota $C$.

Untuk mengecek injektivitas, misalkan
\[
f(n_1) = f(n_2).
\]
Maka:
\[
2n_1 = 2n_2 \Rightarrow n_1 = n_2.
\]

Dengan demikian, $f$ adalah fungsi injektif dari $\mathbb{N}$ ke $C$, sehingga:
\[
B \preceq C.
\]

\vspace{8pt}
\noindent
Secara kardinalitas:
\[
|\mathbb{N}| \le |C|.
\]

\section*{Soal 2. Periksa apakah $A = \left\{ \frac{1}{n} \mid n \in \mathbb{N} \right\}$. merupakan himpunan terbilang?
}

\section*{Jawaban}
\subsection*{Menunjukkan bahwa $A = \left\{ \frac{1}{n} \mid n \in \mathbb{N} \right\}$ terbilang}

Definisikan himpunan:
\[
A = \left\{1, \frac12, \frac13, \frac14, \dots\right\}.
\]

Tentukan fungsi berikut:
\[
g : \mathbb{N} \rightarrow A,\qquad g(n) = \frac{1}{n}.
\]

\paragraph{Injektivitas:}
Misalkan
\[
g(n_1) = g(n_2)
\]
artinya
\[
\frac{1}{n_1} = \frac{1}{n_2} \Rightarrow n_1 = n_2.
\]
Jadi $g$ injektif.

\paragraph{Surjektivitas:}
Ambil sebarang $a \in A$. Maka ia berbentuk:
\[
a = \frac{1}{n}
\]
untuk suatu $n \in \mathbb{N}$.
Sehingga:
\[
g(n) = a,
\]
maka $g$ surjektif.

\paragraph{Kesimpulan:}
Karena $g$ injektif dan surjektif, maka $g$ bijektif dan terdapat korespondensi satu-satu antara $\mathbb{N}$ dan $A$. Oleh karena itu:
\[
|A| = |\mathbb{N}| = \aleph_0
\]
yang berarti $A$ merupakan himpunan \textbf{terbilang}.

\end{document}