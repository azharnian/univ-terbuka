\documentclass[12pt]{article}

\usepackage[margin=0.5in]{geometry}
\usepackage{amsmath, amssymb}
\usepackage{hyperref}
\renewcommand{\arraystretch}{1.6}

\newcommand{\psetheader}[4]{%
  \noindent
  \begin{minipage}[t]{0.6\textwidth}
    #1\\
    #2\\
    #3
  \end{minipage}%
  \hfill
  \begin{minipage}[t]{0.35\textwidth}
    \raggedleft #4
  \end{minipage}

  \vspace{0.5\baselineskip}
  \hrule
  \vspace{1em}
}

\title{Aljabar Elementer - Diskusi Sesi 6}
\author{Anas Azhar \\ FST - Matematika - 056413438 \\ Universitas Terbuka}
\date{\today}


\begin{document}

\psetheader
  {\textit{Aljabar Elementer} - Diskusi Sesi 6}
  {Universitas Terbuka, FST - Matematika, 2025.1}
  {Anas Azhar (056413438)}
  {Minggu, 16 November 2025}

\section*{Soal : Berilah contoh sebuah matriks $3 \times 3 $, kemudian carilah inversnya menggunakan adjoint.}

\section*{Solusi :}
\noindent Matriks asal yang akan dicari inversnya adalah
\[
A =
\begin{bmatrix}
1 & \dfrac{1}{2} & 0\\[4pt]
0 & 1 & \dfrac{1}{3}\\[4pt]
0 & 0 & \dfrac{1}{2}
\end{bmatrix}.
\]

\noindent \textbf{1. Determinan $\det(A)$ dengan metode Sarrus, untuk matriks umum}
\[
A = \begin{bmatrix}
a & b & c\\
d & e & f\\
g & h & i
\end{bmatrix},
\]
\noindent metode Sarrus menyatakan bahwa
\[
\det(A) = aei + bfg + cdh - ceg - bdi - afh.
\]

\noindent Pada matriks, elemen-elemennya adalah
\[
a = 1,\quad b = \dfrac{1}{2},\quad c = 0,\quad
d = 0,\quad e = 1,\quad f = \dfrac{1}{3},\quad
g = 0,\quad h = 0,\quad i = \dfrac{1}{2}.
\]

\noindent Maka
\[
\begin{aligned}
\det(A)
&= aei + bfg + cdh - ceg - bdi - afh\\[4pt]
&= 1\cdot 1\cdot \dfrac{1}{2}
+ \dfrac{1}{2}\cdot \dfrac{1}{3}\cdot 0
+ 0\cdot 0\cdot 0\\[4pt]
&\quad - \Big(
0\cdot 1\cdot 0
+ \dfrac{1}{2}\cdot 0\cdot \dfrac{1}{2}
+ 1\cdot \dfrac{1}{3}\cdot 0
\Big)\\[4pt]
&= \dfrac{1}{2} + 0 + 0 - (0+0+0)\\[4pt]
&= \dfrac{1}{2}.
\end{aligned}
\]


\noindent \textbf{2. Menentukan setiap kofaktor $C_{ij}$, minor $M_{ij}$ yang merupakan determinan submatriks yang diperoleh dengan
menghapus baris ke-$i$ dan kolom ke-$j$ dari $A$.
Kofaktor didefinisikan sebagai}
\[
C_{ij} = (-1)^{i+j} M_{ij}.
\]

\noindent Berikut perhitungan \textbf{setiap} minor dan kofaktor:

\noindent \textbf{Baris pertama}

\paragraph{$C_{11}$}
Hapus baris 1 dan kolom 1:
\[
M_{11} = 
\begin{vmatrix}
1 & \dfrac{1}{3}\\[4pt]
0 & \dfrac{1}{2}
\end{vmatrix}
= 1\cdot \dfrac{1}{2} - \dfrac{1}{3}\cdot 0
= \dfrac{1}{2}.
\]
\[
C_{11} = (-1)^{1+1} M_{11} = (+1)\cdot \dfrac{1}{2} = \dfrac{1}{2}.
\]

\paragraph{$C_{12}$}
Hapus baris 1 dan kolom 2:
\[
M_{12} = 
\begin{vmatrix}
0 & \dfrac{1}{3}\\[4pt]
0 & \dfrac{1}{2}
\end{vmatrix}
= 0\cdot \dfrac{1}{2} - \dfrac{1}{3}\cdot 0 = 0.
\]
\[
C_{12} = (-1)^{1+2} M_{12} = -0 = 0.
\]

\paragraph{$C_{13}$}
Hapus baris 1 dan kolom 3:
\[
M_{13} =
\begin{vmatrix}
0 & 1\\[4pt]
0 & 0
\end{vmatrix}
= 0\cdot 0 - 1\cdot 0 = 0.
\]
\[
C_{13} = (-1)^{1+3} M_{13} = (+1)\cdot 0 = 0.
\]

\noindent \textbf{Baris kedua}

\paragraph{$C_{21}$}
Hapus baris 2 dan kolom 1:
\[
M_{21} =
\begin{vmatrix}
\dfrac{1}{2} & 0\\[4pt]
0 & \dfrac{1}{2}
\end{vmatrix}
= \dfrac{1}{2}\cdot \dfrac{1}{2} - 0\cdot 0
= \dfrac{1}{4}.
\]
\[
C_{21} = (-1)^{2+1} M_{21} = -\,\dfrac{1}{4}.
\]

\paragraph{$C_{22}$}
Hapus baris 2 dan kolom 2:
\[
M_{22} =
\begin{vmatrix}
1 & 0\\[4pt]
0 & \dfrac{1}{2}
\end{vmatrix}
= 1\cdot \dfrac{1}{2} - 0\cdot 0
= \dfrac{1}{2}.
\]
\[
C_{22} = (-1)^{2+2} M_{22} = (+1)\cdot \dfrac{1}{2} = \dfrac{1}{2}.
\]

\paragraph{$C_{23}$}
Hapus baris 2 dan kolom 3:
\[
M_{23} =
\begin{vmatrix}
1 & \dfrac{1}{2}\\[4pt]
0 & 0
\end{vmatrix}
= 1\cdot 0 - \dfrac{1}{2}\cdot 0 = 0.
\]
\[
C_{23} = (-1)^{2+3} M_{23} = -0 = 0.
\]

\noindent \textbf{Baris ketiga}

\paragraph{$C_{31}$}
Hapus baris 3 dan kolom 1:
\[
M_{31} =
\begin{vmatrix}
\dfrac{1}{2} & 0\\[4pt]
1 & \dfrac{1}{3}
\end{vmatrix}
= \dfrac{1}{2}\cdot \dfrac{1}{3} - 0\cdot 1
= \dfrac{1}{6}.
\]
\[
C_{31} = (-1)^{3+1} M_{31} = (+1)\cdot \dfrac{1}{6} = \dfrac{1}{6}.
\]

\paragraph{$C_{32}$}
Hapus baris 3 dan kolom 2:
\[
M_{32} =
\begin{vmatrix}
1 & 0\\[4pt]
0 & \dfrac{1}{3}
\end{vmatrix}
= 1\cdot \dfrac{1}{3} - 0\cdot 0
= \dfrac{1}{3}.
\]
\[
C_{32} = (-1)^{3+2} M_{32} = -\,\dfrac{1}{3}.
\]

\paragraph{$C_{33}$}
Hapus baris 3 dan kolom 3:
\[
M_{33} =
\begin{vmatrix}
1 & \dfrac{1}{2}\\[4pt]
0 & 1
\end{vmatrix}
= 1\cdot 1 - \dfrac{1}{2}\cdot 0
= 1.
\]
\[
C_{33} = (-1)^{3+3} M_{33} = (+1)\cdot 1 = 1.
\]

\noindent \textbf{Matriks kofaktor dan adjoint}

\noindent Dari semua $C_{ij}$ di atas, matriks kofaktor adalah
\[
C =
\begin{bmatrix}
C_{11} & C_{12} & C_{13}\\[4pt]
C_{21} & C_{22} & C_{23}\\[4pt]
C_{31} & C_{32} & C_{33}
\end{bmatrix}
=
\begin{bmatrix}
\dfrac{1}{2} & 0 & 0\\[4pt]
-\dfrac{1}{4} & \dfrac{1}{2} & 0\\[4pt]
\dfrac{1}{6} & -\dfrac{1}{3} & 1
\end{bmatrix}.
\]

\noindent Adjoint dari $A$ adalah transpose dari matriks kofaktor:
\[
\operatorname{adj}(A) = C^{T} =
\begin{bmatrix}
\dfrac{1}{2} & -\dfrac{1}{4} & \dfrac{1}{6}\\[4pt]
0 & \dfrac{1}{2} & -\dfrac{1}{3}\\[4pt]
0 & 0 & 1
\end{bmatrix}.
\]

\noindent \textbf{3. Invers matriks dengan adjoint}

\noindent Rumus umum invers dengan adjoint:
\[
A^{-1} = \frac{1}{\det(A)} \operatorname{adj}(A).
\]

\noindent Karena $\det(A) = \dfrac{1}{2}$, maka
\[
A^{-1}
= \frac{1}{\tfrac{1}{2}} \operatorname{adj}(A)
= 2 \operatorname{adj}(A)
= 2
\begin{bmatrix}
\dfrac{1}{2} & -\dfrac{1}{4} & \dfrac{1}{6}\\[4pt]
0 & \dfrac{1}{2} & -\dfrac{1}{3}\\[4pt]
0 & 0 & 1
\end{bmatrix}.
\]

\noindent Sehingga
\[
A^{-1} =
\begin{bmatrix}
1 & -\dfrac{1}{2} & \dfrac{1}{3}\\[4pt]
0 & 1 & -\dfrac{2}{3}\\[4pt]
0 & 0 & 2
\end{bmatrix}.
\]

\noindent \textbf{4. Verifikasi $A A^{-1} = I$}

Sekarang kita buktikan bahwa
\[
A\cdot A^{-1} = I_3.
\]

Hitung
\[
A\cdot A^{-1} =
\begin{bmatrix}
1 & \dfrac{1}{2} & 0\\[4pt]
0 & 1 & \dfrac{1}{3}\\[4pt]
0 & 0 & \dfrac{1}{2}
\end{bmatrix}
\begin{bmatrix}
1 & -\dfrac{1}{2} & \dfrac{1}{3}\\[4pt]
0 & 1 & -\dfrac{2}{3}\\[4pt]
0 & 0 & 2
\end{bmatrix}.
\]

Elemen-elemen hasil kali:
\[
(1,1): 1\cdot 1 + \dfrac{1}{2}\cdot 0 + 0\cdot 0 = 1,
\]
\[
(1,2): 1\cdot\left(-\dfrac{1}{2}\right) + \dfrac{1}{2}\cdot 1 + 0\cdot 0 = 0,
\]
\[
(1,3): 1\cdot \dfrac{1}{3} + \dfrac{1}{2}\cdot\left(-\dfrac{2}{3}\right) + 0\cdot 2 = 0,
\]
\[
(2,1): 0\cdot 1 + 1\cdot 0 + \dfrac{1}{3}\cdot 0 = 0,
\]
\[
(2,2): 0\cdot\left(-\dfrac{1}{2}\right) + 1\cdot 1 + \dfrac{1}{3}\cdot 0 = 1,
\]
\[
(2,3): 0\cdot \dfrac{1}{3} + 1\cdot\left(-\dfrac{2}{3}\right) + \dfrac{1}{3}\cdot 2 = 0,
\]
\[
(3,1): 0\cdot 1 + 0\cdot 0 + \dfrac{1}{2}\cdot 0 = 0,
\]
\[
(3,2): 0\cdot\left(-\dfrac{1}{2}\right) + 0\cdot 1 + \dfrac{1}{2}\cdot 0 = 0,
\]
\[
(3,3): 0\cdot \dfrac{1}{3} + 0\cdot\left(-\dfrac{2}{3}\right) + \dfrac{1}{2}\cdot 2 = 1.
\]

Maka
\[
A\cdot A^{-1} =
\begin{bmatrix}
1 & 0 & 0\\[4pt]
0 & 1 & 0\\[4pt]
0 & 0 & 1
\end{bmatrix} = I_3.
\]

Dengan demikian, $A^{-1}$ yang diperoleh benar merupakan invers dari $A$.
\end{document}