\documentclass[12pt]{article}

\usepackage[margin=0.5in]{geometry}
\usepackage{amsmath, amssymb}
\usepackage{hyperref}
\renewcommand{\arraystretch}{1.6}

% Header pset untuk halaman pertama
\newcommand{\psetheader}[4]{%
  \noindent
  \begin{minipage}[t]{0.6\textwidth}
    #1\\
    #2\\
    #3
  \end{minipage}%
  \hfill
  \begin{minipage}[t]{0.35\textwidth}
    \raggedleft #4
  \end{minipage}

  \vspace{0.5\baselineskip}
  \hrule
  \vspace{1em}
}

\title{Kalkulus Diferensial - Diskusi Sesi 5}
\author{Anas Azhar \ FST - Matematika - 056413438 \ Universitas Terbuka}
\date{\today}

\begin{document}

\psetheader
  {\textit{Kalkulus Diferensial} - Diskusi Sesi 5}
  {Universitas Terbuka, FST - Matematika, 2025.1}
  {Anas Azhar (056413438)}
  {Minggu, 9 November 2025}

\section*{Soal}
Tentukan turunan fungsi berikut menggunakan aturan rantai:
\begin{enumerate}
    \item \( f(x) = \left(\dfrac{x^2+1}{x+2}\right)^3 \)
    \item \( f(x) = \sin^2(x^2 + 4x + 2) \)
\end{enumerate}

\section*{Solusi}

\begin{enumerate}
    \item Misalkan
    \[
        f(x) = \left(u(x)\right)^3, 
        \quad u(x) = \frac{x^2+1}{x+2}.
    \]
    Dengan aturan rantai,
    \[
        f'(x) = 3\left(u(x)\right)^2 \cdot u'(x).
    \]

    Sekarang cari \(u'(x)\) dengan aturan hasil bagi:
    \[
        u(x) = \frac{x^2+1}{x+2}
    \]
    \[
        u'(x) = \frac{(2x)(x+2) - (x^2+1)\cdot 1}{(x+2)^2}
              = \frac{2x(x+2) - (x^2+1)}{(x+2)^2}
              = \frac{2x^2+4x - x^2 - 1}{(x+2)^2}
              = \frac{x^2+4x-1}{(x+2)^2}.
    \]

    Substitusikan kembali:
    \[
        f'(x) = 3\left(\frac{x^2+1}{x+2}\right)^2 \cdot \frac{x^2+4x-1}{(x+2)^2}.
    \]

    Bentuk yang sudah disederhanakan:
    \[
    \boxed{
        f'(x) = 3\frac{(x^2+1)^2(x^2+4x-1)}{(x+2)^4}.
    }
    \]

    \item Tulis fungsi sebagai komposisi:
    \[
        f(x) = \big(\sin(g(x))\big)^2, 
        \quad g(x) = x^2 + 4x + 2.
    \]
    Dengan aturan rantai,
    \[
        f'(x) = 2\sin(g(x))\cos(g(x))\cdot g'(x).
    \]

    Hitung turunan dalamnya:
    \[
        g'(x) = 2x + 4.
    \]

    Maka
    \[
        f'(x) = 2\sin(x^2+4x+2)\cos(x^2+4x+2)\,(2x+4).
    \]
    
    Jika menggunakan identitas \(2\sin a\cos a = \sin(2a)\), dapat juga ditulis:
    
    \[
    \boxed{
        f'(x) = (2x+4)\,\sin\big(2(x^2+4x+2)\big)
              = (2x+4)\,\sin(2x^2+8x+4).
    }
    \]
\end{enumerate}

\end{document}