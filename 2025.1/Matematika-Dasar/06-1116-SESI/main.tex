\documentclass[12pt]{article}

\usepackage[margin=0.5in]{geometry}
\usepackage{amsmath, amssymb}

\usepackage{tikz}
\usepackage{pgfplots}
\pgfplotsset{compat=1.17}

\usepackage{hyperref}
\renewcommand{\arraystretch}{1.6}

\usepackage{hyperref}
\renewcommand{\arraystretch}{1.6}

\newcommand{\psetheader}[4]{%
  \noindent
  \begin{minipage}[t]{0.6\textwidth}
    #1\\
    #2\\
    #3
  \end{minipage}%
  \hfill
  \begin{minipage}[t]{0.35\textwidth}
    \raggedleft #4
  \end{minipage}

  \vspace{0.5\baselineskip}
  \hrule
  \vspace{1em}
}

\title{Matematika Dasar - Diskusi Sesi 6}
\author{Anas Azhar \\ FST - Matematika - 056413438 \\ Universitas Terbuka}
\date{\today}


\begin{document}

\psetheader
  {\textit{Matematika Dasar} - Diskusi Sesi 6}
  {Universitas Terbuka, FST - Matematika, 2025.1}
  {Anas Azhar (056413438)}
  {Minggu, 16 November 2025}

\section*{Soal 1}

\textbf{a.  $f(x) = x^2 + 2$}

\textbf{Domain:} \\
Fungsi ini adalah polinom, sehingga terdefinisi untuk semua bilangan real:
\[
\text{Domain } f = (-\infty, \infty).
\]

\textbf{Range:} \\
Karena \(x^2 \ge 0\), nilai minimum fungsi adalah:
\[
f_{\min} = 0 + 2 = 2.
\]
Dan untuk \(x \to \infty\), \(f(x) \to \infty\). Jadi:
\[
\text{Range } f = [2,\infty).
\]

\textbf{b. $g(x) = \sqrt{x - 3}$}

\textbf{Domain:} \\
Agar akar terdefinisi, harus berlaku:
\[
x - 3 \ge 0 \quad \Rightarrow \quad x \ge 3.
\]
Sehingga:
\[
\text{Domain } g = [3,\infty).
\]

\textbf{Range:} \\
Nilai akar tidak pernah negatif. Untuk \(x = 3\):
\[
g(3) = \sqrt{0} = 0.
\]
Jika \(x\) semakin besar, nilai \(g(x)\) semakin besar:
\[
\text{Range } g = [0,\infty).
\]

\section*{Soal 2}

Dimisalkan fungsi
\[
y =
\begin{cases}
x^2 - 1, & x < 0,\\[4pt]
\sqrt{x}, & x \ge 0.
\end{cases}
\]

Untuk menggambar grafik fungsi tersebut, perhatikan dua bagian berikut.

\begin{enumerate}
  \item Untuk $x < 0$ berlaku $y = x^2 - 1$.\\
        Grafik $y = x^2 - 1$ adalah parabola yang terbuka ke atas dengan puncak di $(0,-1)$.
        Namun karena syaratnya $x<0$, yang digambar hanya \emph{setengah kiri} parabola.
        Titik $(0,-1)$ \textbf{tidak} termasuk, sehingga digambar sebagai lingkaran kosong.

  \item Untuk $x \ge 0$ berlaku $y = \sqrt{x}$.\\
        Grafik $y = \sqrt{x}$ adalah kurva akar yang mulai dari titik $(0,0)$ dan 
        menuju ke kanan. Karena $x \ge 0$, titik $(0,0)$ \textbf{termasuk} dalam grafik
        (digambar sebagai titik penuh).
\end{enumerate}

Dengan demikian, grafik fungsi terdiri dari:
\begin{itemize}
  \item setengah parabola $y = x^2 - 1$ di sisi kiri sumbu-$y$ untuk $x<0$,
        dengan lingkaran kosong di titik $(0,-1)$;
  \item kurva akar $y = \sqrt{x}$ di sisi kanan sumbu-$y$ untuk $x \ge 0$,
        dengan titik penuh di $(0,0)$.
\end{itemize}

% Opsional: sketsa grafik dengan TikZ
\begin{center}
\begin{tikzpicture}[scale=0.8]
  % Sumbu
  \draw[->] (-4,0) -- (4,0) node[right] {$x$};
  \draw[->] (0,-2) -- (0,4) node[above] {$y$};

  % y = x^2 - 1, x<0
  \draw[domain=-3:0, smooth, samples=80] plot (\x,{(\x)^2 - 1});
  % titik kosong di (0,-1)
  \draw (0,-1) circle (2pt);

  % y = sqrt(x), x>=0
  \draw[domain=0:4, smooth, samples=80] plot (\x,{sqrt(\x)});
  % titik penuh di (0,0)
  \fill (0,0) circle (2pt);
\end{tikzpicture}
\end{center}

\section*{Soal 3}

\textbf{Diberikan fungsi:}
\[
f(x) = \frac{1}{2}x^{3} + 2.
\]

\textbf{1. Memeriksa apakah fungsi satu-satu (injective).}

Turunan dari fungsi adalah
\[
f'(x) = \frac{3}{2}x^{2}.
\]

Karena $f'(x) \ge 0$ untuk semua $x$ dan hanya bernilai $0$ pada $x=0$, maka fungsi selalu meningkat (monoton naik). Fungsi yang monoton naik pada seluruh domainnya merupakan fungsi satu-satu.

\[
\boxed{f(x)\ \text{adalah fungsi satu-satu (injective).}}
\]

\textbf{2. Memeriksa apakah fungsi pada (surjective).}

Karena fungsi kubik memetakan seluruh $\mathbb{R}$ ke $\mathbb{R}$ dan penambahan serta perkalian konstanta tidak mengubah sifat tersebut, maka
\[
\lim_{x \to -\infty} f(x) = -\infty, \qquad 
\lim_{x \to \infty} f(x) = \infty.
\]

Dengan demikian range fungsi adalah seluruh bilangan real.

\[
\boxed{f(x)\ \text{adalah fungsi pada (surjective) terhadap}\ \mathbb{R}.}
\]

\textbf{3. Kesimpulan.}

Karena fungsi tersebut bersifat satu-satu dan pada, maka
\[
\boxed{f(x)\ \text{merupakan fungsi bijektif (satu-satu dan pada).}}
\]

\section*{Soal 4}

Diketahui
\[
f(x) = x + 8, \qquad g(x) = \sqrt{x + 1}.
\]

\textbf{a. Menentukan } $(f \circ g)(x)$
\[
(f \circ g)(x) = f(g(x)).
\]
Karena $g(x) = \sqrt{x + 1}$, maka
\[
f(g(x)) = f\big(\sqrt{x+1}\big)
= \sqrt{x+1} + 8.
\]
Jadi,
\[
\boxed{(f \circ g)(x) = \sqrt{x+1} + 8}.
\]

\textbf{b. Menentukan } $(g \circ f)(x)$
\[
(g \circ f)(x) = g(f(x)).
\]
Karena $f(x) = x + 8$, maka
\[
g(f(x)) = g(x+8)
= \sqrt{(x+8)+1}
= \sqrt{x + 9}.
\]
Sehingga,
\[
\boxed{(g \circ f)(x) = \sqrt{x + 9}}.
\]

\end{document}