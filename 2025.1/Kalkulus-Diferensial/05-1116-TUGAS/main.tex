\documentclass[12pt]{article}

\usepackage[margin=0.5in]{geometry}
\usepackage{amsmath, amssymb}
\usepackage{hyperref}
\renewcommand{\arraystretch}{1.6}

% Header pset untuk halaman pertama
\newcommand{\psetheader}[4]{%
  \noindent
  \begin{minipage}[t]{0.6\textwidth}
    #1\\
    #2\\
    #3
  \end{minipage}%
  \hfill
  \begin{minipage}[t]{0.35\textwidth}
    \raggedleft #4
  \end{minipage}

  \vspace{0.5\baselineskip}
  \hrule
  \vspace{1em}
}

\title{Kalkulus Diferensial - Tugas 2}
\author{Anas Azhar \ FST - Matematika - 056413438 \ Universitas Terbuka}
\date{\today}

\begin{document}

\psetheader
  {\textit{Kalkulus Diferensial} - Tugas 2}
  {Universitas Terbuka, FST - Matematika, 2025.1}
  {Anas Azhar (056413438)}
  {Minggu, 9 November 2025}


\section*{Soal 1}
Jika $y = 3x^2 - 2x + 5$, tentukan $\dfrac{dy}{dx}$ dengan menggunakan definisi turunan.

\noindent \textbf{Penyelesaian}

Misalkan $f(x) = 3x^2 - 2x + 5$. Definisi turunan:
\[
f'(x) = \lim_{h \to 0} \frac{f(x+h) - f(x)}{h}.
\]

Pertama, hitung $f(x+h)$:
\begin{align*}
f(x+h) &= 3(x+h)^2 - 2(x+h) + 5 \\
       &= 3(x^2 + 2xh + h^2) - 2x - 2h + 5 \\
       &= 3x^2 + 6xh + 3h^2 - 2x - 2h + 5.
\end{align*}

Kemudian, hitung selisih $f(x+h) - f(x)$:
\begin{align*}
f(x+h) - f(x)
&= \big(3x^2 + 6xh + 3h^2 - 2x - 2h + 5\big)
   - \big(3x^2 - 2x + 5\big) \\
&= 6xh + 3h^2 - 2h \\
&= h(6x + 3h - 2).
\end{align*}

Substitusikan ke definisi turunan:
\begin{align*}
f'(x) &= \lim_{h \to 0} \frac{f(x+h) - f(x)}{h} \\
      &= \lim_{h \to 0} \frac{h(6x + 3h - 2)}{h} \\
      &= \lim_{h \to 0} (6x + 3h - 2) \\
      &= 6x - 2.
\end{align*}

Jadi,
\[
\frac{dy}{dx} = f'(x) = 6x - 2.
\]

\section*{Soal 2}
\textbf{Diketahui:} 
\[
f(x) = x^2 - 1
\]

\textbf{Ditanya:}
\begin{enumerate}
    \item Persamaan garis singgung di \(x = -1\);
    \item Persamaan garis normal di \(x = -1\).
\end{enumerate}

\noindent \textbf{Penyelesaian}

Pertama, tentukan nilai fungsi pada titik tersebut:
\[
f(-1) = (-1)^2 - 1 = 0.
\]
Jadi titik yang dilalui garis singgung dan normal adalah
\[
(-1,\,0).
\]

Selanjutnya, cari turunan fungsi:
\[
f'(x) = 2x.
\]
Maka gradien garis singgung pada \(x = -1\):
\[
m_{\text{tangent}} = f'(-1) = 2(-1) = -2.
\]

\textbf{a. Garis Singgung}

Gunakan persamaan garis:
\[
y - y_1 = m(x - x_1),
\]
dengan titik \((-1, 0)\) dan gradien \(-2\):

\[
y - 0 = -2(x + 1)
\]
\[
y = -2x - 2.
\]

\textbf{b. Garis Normal}

Gradien garis normal adalah negatif kebalikan dari gradien garis singgung:
\[
m_{\text{normal}} = -\frac{1}{m_{\text{tangent}}}
= -\frac{1}{-2} = \frac{1}{2}.
\]

Maka persamaan garis normal:
\[
y - 0 = \frac{1}{2}(x + 1)
\]
\[
y = \frac{1}{2}x + \frac{1}{2}.
\]

\textbf{Jadi:}
\[
\text{Garis singgung: } y = -2x - 2
\]
\[
\text{Garis normal: } y = \frac{1}{2}x + \frac{1}{2}
\]

\section*{Soal 3}
\textbf{Diketahui:}
\[
y = x \cos(2x)
\]

\noindent \textbf{Ditanya:}
\begin{enumerate}
    \item $y'''(x)$
    \item $y'''(\pi)$
\end{enumerate}

\noindent \textbf{Penyelesaian}

Turunan pertama menggunakan aturan perkalian:
\[
y' = \cos(2x) - 2x\sin(2x).
\]

Turunan kedua:
\[
y'' = -4\sin(2x) - 4x\cos(2x).
\]

Turunan ketiga:
\begin{align*}
y''' &= (-4\sin 2x)' + (-4x\cos 2x)' \\
     &= -8\cos(2x) - 4\cos(2x) + 8x\sin(2x) \\
     &= -12\cos(2x) + 8x\sin(2x).
\end{align*}

\[
\boxed{y'''(x) = -12\cos(2x) + 8x\sin(2x)}
\]

Evaluasi di $x = \pi$:
\[
y'''(\pi)
= -12\cos(2\pi) + 8\pi\sin(2\pi)
= -12(1) + 0
= -12.
\]

\[
\boxed{y'''(\pi) = -12}
\]

\section*{Soal 4}

Tentukan $\dfrac{dy}{dx}$ dari fungsi:
\begin{enumerate}
    \item $y = \ln(\sin(x^3 - 1))$; [menggunakan aturan rantai]
    \item Fungsi implisit: $2x^2y - \sin(xy^2) + e^{-xy} = 10$.
\end{enumerate}

\noindent \textbf{Penyelesaian}

\begin{enumerate}
    \item[\textbf{a.}] Diketahui
    \[
    y = \ln(\sin(x^3 - 1)).
    \]
    Misalkan
    \[
    u = \sin(x^3 - 1),
    \quad \text{sehingga} \quad y = \ln u.
    \]

    Turunan berantai:
    \[
    \frac{dy}{dx} = \frac{dy}{du}\cdot\frac{du}{dx}.
    \]

    Pertama,
    \[
    \frac{dy}{du} = \frac{1}{u} = \frac{1}{\sin(x^3-1)}.
    \]

    Kedua, turunkan $u$ terhadap $x$:
    \[
    u = \sin(x^3 - 1) \Rightarrow
    \frac{du}{dx} = \cos(x^3 - 1)\cdot 3x^2.
    \]

    Maka
    \begin{align*}
    \frac{dy}{dx}
      &= \frac{1}{\sin(x^3-1)} \cdot 3x^2\cos(x^3-1) \\
      &= \frac{3x^2\cos(x^3-1)}{\sin(x^3-1)} \\
      &= 3x^2\cot(x^3-1).
    \end{align*}

    Jadi,
    \[
    \boxed{\dfrac{dy}{dx} = \dfrac{3x^2\cos(x^3-1)}{\sin(x^3-1)} = 3x^2\cot(x^3-1)}.
    \]

    \item[\textbf{b.}] Diberikan fungsi implisit
    \[
    2x^2y - \sin(xy^2) + e^{-xy} = 10.
    \]

    Turunkan kedua ruas terhadap $x$ (ingat bahwa $y = y(x)$, sehingga muncul $y'$):

    \[
    \frac{d}{dx}\big(2x^2y\big)
    - \frac{d}{dx}\big(\sin(xy^2)\big)
    + \frac{d}{dx}\big(e^{-xy}\big)
    = \frac{d}{dx}(10).
    \]

    Hitung satu per satu:

    \[
    \frac{d}{dx}(2x^2y)
    = 4xy + 2x^2y'
    \quad\text{(aturan perkalian)}.
    \]

    \[
    \frac{d}{dx}(\sin(xy^2))
    = \cos(xy^2)\cdot\frac{d}{dx}(xy^2).
    \]
    Dan
    \[
    \frac{d}{dx}(xy^2)
    = y^2 + x\cdot 2y y' = y^2 + 2xyy'.
    \]
    Maka
    \[
    \frac{d}{dx}(\sin(xy^2))
    = \cos(xy^2)\big(y^2 + 2xyy'\big).
    \]

    Selanjutnya,
    \[
    \frac{d}{dx}(e^{-xy})
    = e^{-xy}\cdot\frac{d}{dx}(-xy)
    = e^{-xy}\cdot(-(y + xy'))
    = -(y + xy')e^{-xy}.
    \]

    Substitusikan ke persamaan turunan:
    \begin{align*}
    4xy + 2x^2y'
    &- \cos(xy^2)\big(y^2 + 2xyy'\big)
    -(y + xy')e^{-xy}
    = 0.
    \end{align*}

    Kelompokkan suku-suku yang mengandung $y'$ dan yang tidak:

    \[
    4xy 
    - y^2\cos(xy^2)
    - y e^{-xy}
    + \big(2x^2y' - 2xyy'\cos(xy^2) - xy'e^{-xy}\big)
    = 0.
    \]

    Faktorkan $y'$:
    \[
    4xy - y^2\cos(xy^2) - y e^{-xy}
    + y'\big(2x^2 - 2xy\cos(xy^2) - x e^{-xy}\big)
    = 0.
    \]

    Pindahkan suku tanpa $y'$ ke ruas kanan:
    \[
    y'\big(2x^2 - 2xy\cos(xy^2) - x e^{-xy}\big)
    = -4xy + y^2\cos(xy^2) + y e^{-xy}.
    \]

    Sehingga
    \[
    y' = \frac{-4xy + y^2\cos(xy^2) + y e^{-xy}}
              {2x^2 - 2xy\cos(xy^2) - x e^{-xy}}.
    \]

    Dapat juga difaktorkan:
    \[
    y' = \frac{y\big(-4x + y\cos(xy^2) + e^{-xy}\big)}
              {x\big(2x - 2y\cos(xy^2) - e^{-xy}\big)}.
    \]

    Jadi,
    \[
    \boxed{\dfrac{dy}{dx} = 
    \frac{-4xy + y^2\cos(xy^2) + y e^{-xy}}
         {2x^2 - 2xy\cos(xy^2) - x e^{-xy}}
    }.
    \]
\end{enumerate}

\section*{Soal 5}
\textbf{Diketahui:} 
\[
f(x) = \sin(2x).
\]

\noindent \textbf{Ditanya:}
\begin{enumerate}
    \item Deret Taylor di sekitar \(x = \dfrac{\pi}{2}\), sampai 5 suku.
    \item Deret Maclaurin (sekitar \(x=0\)), sampai 4 suku.
\end{enumerate}

\noindent \textbf{Penyelesaian}

\section*{a. Deret Taylor di sekitar \(x = \dfrac{\pi}{2}\)}

Rumus umum deret Taylor fungsi \(f(x)\) di sekitar \(x=a\):
\[
f(x) = \sum_{n=0}^{\infty} \frac{f^{(n)}(a)}{n!}(x-a)^n.
\]

Ambil turunan berurutan dari \(f(x)=\sin(2x)\):

\[
\begin{aligned}
f(x) &= \sin(2x), \\
f'(x) &= 2\cos(2x), \\
f''(x) &= -4\sin(2x), \\
f'''(x) &= -8\cos(2x), \\
f^{(4)}(x) &= 16\sin(2x), \\
f^{(5)}(x) &= 32\cos(2x), \\
&\;\;\vdots
\end{aligned}
\]

Nilai turunan di \(a = \dfrac{\pi}{2}\) (ingat \(\sin \pi = 0\), \(\cos \pi = -1\)):

\[
\begin{aligned}
f(a) &= \sin(\pi) = 0, \\
f'(a) &= 2\cos(\pi) = -2, \\
f''(a) &= -4\sin(\pi) = 0, \\
f'''(a) &= -8\cos(\pi) = 8, \\
f^{(4)}(a) &= 16\sin(\pi) = 0, \\
f^{(5)}(a) &= 32\cos(\pi) = -32, \\
f^{(7)}(a) &= 128\cos(\pi) = 128, \\
f^{(9)}(a) &= 512\cos(\pi) = -512.
\end{aligned}
\]

Masukkan ke rumus Taylor:
\[
\begin{aligned}
\sin(2x)
&= \frac{f'(a)}{1!}(x-a) + \frac{f'''(a)}{3!}(x-a)^3
   + \frac{f^{(5)}(a)}{5!}(x-a)^5
   + \frac{f^{(7)}(a)}{7!}(x-a)^7
   + \frac{f^{(9)}(a)}{9!}(x-a)^9 + \cdots \\
&= -2(x-a) + \frac{8}{3!}(x-a)^3 - \frac{32}{5!}(x-a)^5
   + \frac{128}{7!}(x-a)^7 - \frac{512}{9!}(x-a)^9 + \cdots \\
&= -2(x-a) + \frac{4}{3}(x-a)^3 - \frac{4}{15}(x-a)^5
   + \frac{8}{315}(x-a)^7 - \frac{4}{2835}(x-a)^9 + \cdots,
\end{aligned}
\]
dengan \(a = \dfrac{\pi}{2}\).

Jadi deret Taylor di sekitar \(x = \dfrac{\pi}{2}\) sampai 5 suku (tak nol) adalah
\[
\boxed{
\sin(2x)
= -2\Big(x-\frac{\pi}{2}\Big)
+ \frac{4}{3}\Big(x-\frac{\pi}{2}\Big)^3
- \frac{4}{15}\Big(x-\frac{\pi}{2}\Big)^5
+ \frac{8}{315}\Big(x-\frac{\pi}{2}\Big)^7
- \frac{4}{2835}\Big(x-\frac{\pi}{2}\Big)^9
+ \cdots }.
\]

\section*{b. Deret Maclaurin (sekitar \(x=0\))}

Deret Maclaurin adalah deret Taylor dengan \(a = 0\):
\[
f(x) = \sum_{n=0}^{\infty} \frac{f^{(n)}(0)}{n!}x^n.
\]

Cara cepat: gunakan deret Maclaurin dari \(\sin z\),
\[
\sin z = z - \frac{z^3}{3!} + \frac{z^5}{5!} - \frac{z^7}{7!} + \cdots
\]
dengan \(z = 2x\):
\[
\begin{aligned}
\sin(2x)
&= 2x - \frac{(2x)^3}{3!} + \frac{(2x)^5}{5!} - \frac{(2x)^7}{7!} + \cdots \\
&= 2x - \frac{8x^3}{6} + \frac{32x^5}{120} - \frac{128x^7}{5040} + \cdots \\
&= 2x - \frac{4}{3}x^3 + \frac{4}{15}x^5 - \frac{8}{315}x^7 + \cdots.
\end{aligned}
\]

Jadi deret Maclaurin sampai 4 suku:
\[
\boxed{
\sin(2x) = 2x - \frac{4}{3}x^3 + \frac{4}{15}x^5 - \frac{8}{315}x^7 + \cdots }.
\]

\end{document}